\pdfoutput=1
\pdfcompresslevel=9
\pdfinfo
{
    /Author (Jacek Witkowski)
    /Title (Metody głębokiego uczenia w wybranych problemach klasyfikacji.)
    /Subject (Praca magisterska)
    /Keywords (dcnn cnn splot rbm głębokie sieci neuronowe deep learning)
}

\documentclass[a4paper,onecolumn,oneside,12pt,wide,floatssmall]{mwrep}

\usepackage{listings}
\usepackage[framemethod=tikz]{mdframed}

\lstset{
    inputencoding=utf8x,
    extendedchars=\true,
    literate={ą}{{\k{a}}}1
             {Ą}{{\k{A}}}1
             {ę}{{\k{e}}}1
             {Ę}{{\k{E}}}1
             {ó}{{\'o}}1
             {Ó}{{\'O}}1
             {ś}{{\'s}}1
             {Ś}{{\'S}}1
             {ł}{{\l{}}}1
             {Ł}{{\L{}}}1
             {ż}{{\.z}}1
             {Ż}{{\.Z}}1
             {ź}{{\'z}}1
             {Ź}{{\'Z}}1
             {ć}{{\'c}}1
             {Ć}{{\'C}}1
             {ń}{{\'n}}1
             {Ń}{{\'N}}1
}
% Default fixed font does not support bold face
\DeclareFixedFont{\ttb}{T1}{txtt}{bx}{n}{11} % for bold
\DeclareFixedFont{\ttm}{T1}{txtt}{m}{n}{11}  % for normal

% Custom colors
\usepackage{color}
\definecolor{deepblue}{rgb}{0,0,0.5}
\definecolor{deepred}{rgb}{0.6,0,0}
\definecolor{deepgreen}{rgb}{0,0.5,0}
\definecolor{pycharm_background}{rgb}{0.17,0.17,0.17}
\definecolor{pycharm_keywords}{rgb}{0.92,0.46,0}
\definecolor{pycharm_code}{rgb}{0.66,0.71,0.78}
\definecolor{pycharm_comment}{rgb}{0.6,0.6,0.6}


% Python style for highlighting
\newcommand\pythonstyle{\lstset{
language=Python,
basicstyle=\linespread{0.94}\ttm\color{pycharm_code},
otherkeywords={self, as},             % Add keywords here
keywordstyle=\linespread{0.94}\ttb\color{pycharm_keywords},
emph={MyClass,__init__},          % Custom highlighting
emphstyle=\linespread{0.94}\ttb\color{deepred},    % Custom highlighting style
stringstyle=\color{deepgreen},
frame=none,
showstringspaces=false,
%backgroundcolor = \color{pycharm_background},
commentstyle=\linespread{0.94}\ttm\color{pycharm_comment},
breakatwhitespace=false,
breaklines=true
}}


% Python environment
\lstnewenvironment{python}[1][]
{
\pythonstyle
\lstset{#1}
}
{}

% Python for external files
\newcommand\pythonexternal[2][]{{
\pythonstyle
\begin{mdframed}[backgroundcolor=pycharm_background, hidealllines=true]
    \lstinputlisting[#1]{#2}
\end{mdframed}
}}


% basic packages
\usepackage{float}
\usepackage{polski}
\usepackage[export]{adjustbox} % ramki dla obrazków
\usepackage{pgfplots} % wykresy
\usepackage{amsmath}
\usepackage{amsfonts}
\usepackage{graphicx}
\usepackage{parskip}
\usepackage{enumitem}
\usepackage[utf8x]{inputenc}
\usepackage{fullpage}

\newcommand\defeq{\mathrel{\overset{\makebox[0pt]{\mbox{\normalfont\tiny\sffamily def}}}{=}}}


% bibliography and links
\usepackage{url}
\usepackage[noadjust]{cite}
\usepackage{multicol}
\def\UrlBreaks{\do\/\do-}
\usepackage[hidelinks]{hyperref}

% graphs
\usepackage{tikz}
\usetikzlibrary{arrows}

\setlist[itemize]{label=\textbullet}

\newenvironment{Figure}
  {\par\medskip\noindent\minipage{\linewidth}}
  {\endminipage\par\medskip}
  
\begin{document}
\begin{titlepage}
  \begin{center}

    \begin{figure}[H]
	\centering
	\includegraphics[width=0.8\linewidth]{img/elka-logo.png}
    \end{figure}


    \textsc{\Large Instytut Informatyki}\\[4.0cm]

    \begin{figure}[H]
	\centering
	\includegraphics[width=0.8\linewidth]{img/pr-dyplomowa-text.png}
    \end{figure}
    \normalsize
    \begin{tabular}{rl}
      na~kierunku:& Informatyka \\
      w~specjalności:& Inżynieria Systemów Informacyjnych
     \end{tabular} \\[1.5cm]

    \Large Metody głębokiego uczenia w~wybranych problemach klasyfikacji
    \vfill
    \Huge Jacek Witkowski \\
    \normalsize nr albumu: 236735 \\[0.5cm]

    \normalsize \textit{promotor}\\
    mgr inż. Rajmund Kożuszek
    \vfill

    {\large Warszawa 2017}

  \end{center}
\end{titlepage}

\tableofcontents

\chapter{Wstęp}

\section{Cel pracy}
Celem niniejszej pracy magisterskiej jest zbadanie wpływu różnych zabiegów
stosowanych w~splotowych sieciach neuronowych na~jakość działania owych sieci.
W~pracy skupiono się~wyłącznie na~aspektach dotyczących jakości klasyfikacji,
w~szczególności~na~dokładności (\textit{ang.~accuracy}). Aspekty dotyczące złożoności
obliczeniowej oraz złożoności pamięciowej celowo zostały pominięte w~pracy.

\section{Historia uczenia maszynowego}
W znakomitej większości przypadków tworzenie programu komputerowego polega
na podaniu ciągu instrukcji, które mają zostać wykonane, aby rozwiązać zadany
problem. Co~jednak, jeśli nie wiadomo, w jaki sposób osiągnąć zamierzony cel?

W 1956 roku Arthur Samuel postanowił napisać program komputerowy
potrafiący grać w~warcaby lepiej niż jego autor. Stworzył więc aplikację,
która grała sama ze~sobą, ucząc się na własnych błędach i sukcesach.
Z czasem oprogramowanie stawało się coraz to lepsze. W 1962 roku program był
na~tyle dobry, że pokonał mistrza stanu Connecticut: Roberta Nealey'ego
(obecnie istnieje algorytm, z którym nie da się wygrać w warcaby).

Z~czasem uczenie maszynowe powoli zyskiwało na~zainteresowaniu.
Wiele tworzonych mechanizmów było inspirowanych naturą (np. algorytmy
ewolucyjne czy sztuczne sieci neuronowe). Jednak dużym ograniczeniem
w~większości problemów wciąż pozostawała konieczność nadzorowania procesu
uczenia, tzn. dla~każdego zbioru wejściowego należało podać spodziewany wynik
(np. przy rozpoznawaniu cyfr podawano rzeczywistą cyfrę odpowiadającą podanemu
obrazkowi).

W 1986 roku opracowano teoretyczny model mechanizmu potrafiącego stwierdzać
podobieństwo różnych danych i uczyć się bez nadzoru. Była to tzw.~Maszyna
Boltzmanna. Potrafiła ona zauważać cechy charakterystyczne dla~różnego typu
danych (np. w problemie rozpoznawania cyfr zauważała charakterystyczne łuki,
które odróżniały jedne znaki od drugich).
Nie istniał jednak żaden efektywny sposób uczenia mechanizmu.
Dopiero w połowie pierwszej dekady XXI wieku zaczęły powstawać pierwsze szybkie
algorytmy uczące. Wówczas rozwój dziedziny znacznie przyspieszył.
Szczególne zainteresowanie zyskało podejście zwane głębokim uczeniem.

\section{Zastosowania}
Głębokie uczenie (\textit{ang. deep learning}), zwane również uczeniem
o~głębokich strukturach (\textit{ang. deep structured learning}) lub uczeniem
hierarchicznym (\textit{ang. hierarchical learning}), to~koncepcja uczenia
polegająca na~tworzeniu modelu danych o wielu warstwach, gdzie każda kolejna
warstwa reprezentuje wyższy poziom abstrakcji niż poprzednia (warstwa wejściowa
ma~poziom najniższy). Przykładowo dla warstwowej sieci neuronowej rozpoznającej
twarze, w~pierwszej warstwie ukrytej zauważane są~najprostsze cechy (np. łuki,
krawędzie), w kolejnych warstwach rozpoznawane są coraz to~większe fragmenty
obrazka, by~w~warstwie wyjściowej uzyskać rozpoznanie całej twarzy.

Obecnie podejście to~jest z~powodzeniem wykorzystywane w~takich usługach jak:
\begin{itemize}
  \item Google Search by Image,
  \item Google Voice Search,
  \item Google Now,
  \item Siri,
  \item S Voice,
  \item Amazon Recommendations,
  \item Google Self-Driving Car.
\end{itemize}

Szczególnie dużo projektów wykorzystujących głębokie uczenie prowadzi firma
Google (w ramach projektu Google Brain). Między innymi jednym z~problemów,
którym się~zajmowała było oznaczanie na~mapie adresów budynków na~podstawie
obrazów z~systemu Google Street View. Choć~najpierw rozpoznawaniem zajmowali
się~ludzie, to~na początku 2014 roku stworzono system, który
samodzielnie potrafił rozpoznawać adresy. Robił to~znacznie szybciej niż ludzie
(adresy wszystkich budynków we~Francji rozpoznał w niespełna godzinę),
a~przy~tym~popełniał mniej błędów.

W~ramach badań nad~głębokim uczeniem, w~obrębie projektu Microsoft Research,
powstał tłumacz potrafiący w~czasie rzeczywistym tłumaczyć przemówienia
wygłaszane w~języku angielskim na~język chiński, korzystając z~głosu
osoby mówiącej. Innym znacznym osiągnięciem w~dziedzinie przetwarzania języka naturalnego (\textit{ang.
Natural Language Processing, NLP}) był system stworzony przez~IBM: Watson.
Głównym celem systemu było odpowiadanie na~pytania zadawane w~języku naturalnym.
W~2011 roku Watson wziął udział w~amerykańskim teleturnieju \textit{Jeopardy!} (polskim odpowiednikiem był
teleturniej \textit{Va Banque}). Gra polegała na~odpowiadaniu na~pytania z~różnych dziedzin, a~za~każdą
prawidłową odpowiedź gracz otrzymywał punkty. Watson był tak dobry, że~pokonał najlepszych graczy i~wygrał
milion dolarów.

Choć rozpoznawanie obrazów było niegdyś dziedziną, w~której komputery wypadały
znacznie gorzej niż~ludzie, to~wraz z~rozwojem głębokiego uczenia sytuacja
uległa zmianie. Już w~2011 roku podczas konferencji IJCNN (International Joint
Conference on Neural Networks) komputery poradziły sobie z~problemem
rozpoznawania znaków drogowych lepiej od ludzi (dwukrotnie lepiej).

Kolejnym przełomem w~rozpoznawaniu obrazów był~system stworzony w~ramach
Microsoft Research w lutym 2015 roku, którego zadaniem było stwierdzenie,
co~znajduje się~na~obrazku, a~następnie przypisanie mu odpowiedniej kategorii spośród
ponad 100000 dostępnych. Okazało się, że~ludzie średnio popełniali 5.1\% błędów,
a~algorytm Microsoftu: 4.94\%.

Inne zastosowania głębokiego uczenia jakie są badane to m.in:
\begin{itemize}
  \item rozpoznawanie niechcianych wiadomości (tzw. spamu),
  \item automatyczne generowanie opisów dla obrazków w~języku naturalnym (\cite{img-desc-generator}),
  \item rozpoznawanie zmian nowotworowych,
  \item opracowywanie nowych leków,
  \item rozpoznawanie emocji (\textit{ang. affective computing}).
\end{itemize}

\chapter{Sztuczne sieci neuronowe}
W~uczeniu maszynowym poprzez \textbf{sztuczne sieci neuronowe} rozumiana jest rodzina statystycznych modeli
uczenia się zainspirowanych biologicznymi sieciami neuronowymi. Mechanizmy te~są wykorzystywane do~estymacji
lub~aproksymacji funkcji, które zależą od~wielu wartości wejściowych i~są co~do~zasady nieznane.

Sieć neuronowa składa się~z~wielu połączonych ze~sobą jednostek (zwanych neuronami), które wymieniają między
sobą informacje. Połączenia między neuronami posiadają przypisane wagi, które w~trakcie uczenia sieci
są~modyfikowane. Typowo każdy neuron na~wejściu otrzymuje zbiór wartości (pochodzący z~wyjść innych neuronów
lub~będący danymi wejściowymi sieci). Następnie wartość każdego z~wejść jest mnożona przez~wagę przypisaną
do~odpowiedniego wejścia. Tak otrzymana suma jest poddawana działaniu funkcji aktywacji, która jest
charakterystyczna dla~danego typu neuronu i~nie~ulega zmianie wraz z~działaniem sieci.

\section{Model neuronu}
Przykładowym neuronem używanym powszechnie w~sieciach neuronowych jest neuron z~sigmoidalną funkcją aktywacji
(nazywany również neuronem sigmoidalnym). Schemat takiej jednostki przedstawiono na~poniższym rysunku.

\begin{Figure}
	\centering
	\includegraphics[width=\linewidth]{img/sigmoid-neuron.png}
\end{Figure}

W~pierwszej fazie działania neuronu sumowana są wartości na wejściach neuronu pomnożone przez~odpowiadające
im~wagi. W~drugiej fazie dla~otrzymanej sumy obliczana jest wartość funkcji sigmoidalnej
$f(x)=\frac{1}{1+e^{-x}}$, która stanowi wartość wyjściową neuronu.

Innym często stosowanym neuronem jest neuron o progowej funkcji aktywacji, tzn. dla wartości powyżej pewnego
progu przyjmuje wartość 1, a dla pozostałych wartości przyjmuje 0 (lub -1). Wyjścia jednostek tego typu
zazwyczaj są~jednocześnie wyjściami całej sieci neuronowej i~nie~są przesyłane na~wejścia innych neuronów.

\subsection{Funkcje aktywacji}
W~neuronach wykorzystywane są bardzo różne funkcje aktywacji. Każda z~nich zawiera swoje zalety i wady. W~niniejszej
podsekcji przedstawiono i omówiono wybrane funkcje.

\subsubsection{Funkcja liniowa}
Najprostsza z~możliwych funkcji aktywacji. Dokonuje lioniowego przekształcenia na~wartości wejściowej. Jest rzadko
stosowana do~zadań klasyfikacji w~sieciach neuronowych, gdyż~sieć składająca się~z~neuronów o~liniowej~funkcji
aktywacji nie~może modelować nieliniowych funkcji.

\begin{equation*}
f(x)=x
\end{equation*}
\begin{figure}[H]
    \centering
    \begin{tikzpicture}
        \begin{axis}[
        scale only axis, % The height and width argument only apply to the actual axis
        height=5cm,
        xtick={-1,1},
        width=\textwidth,
        axis x line=center, axis y line=center,
        xlabel=$x$,
        ylabel=$f(x)$
        ]
        \addplot [
            domain=-1.1:1.1,
            color=red
        ]{x};
        \end{axis}
    \end{tikzpicture}
    \caption{Funkcja liniowa}
\end{figure}

\subsubsection{Funkcja progowa}
Funkcja nieliniowa, a~więc nadająca się~do~zadań klasyfikacji. Posiada jednak wadę przeszkadzającą w~uczeniu sieci
z~wykorzystaniem metod gradientowych: nie~ma~ciągłej pierwszej pochodniej.

TODO wykres i odnośnik

\subsubsection{Funkcja sigmoidalna}
Również jest to~funkcja nieliniowa. W~przeciwieństwie do~funkcji poprzedniej jest gładka. Jednakże również nie~jest
pozbawiona wad~--~przyczynia się~ona~do~powstawania problemu zanikającego gradientu (TODO cytat), co~wynika z~tego,
że~dla~wysokich wartości wejściowych tej funkcji aktywacji jej gradient jest bliski zera.

TODO wykres i odnośnik

\subsubsection{Poprawiona jednostka liniowa (\textit{ang. Rectified Linear Unit, ReLU})}
Funkcja ta~jest jedną z~najczęściej wykorzystywanych funkcji aktywacji w~nowoczesnych rozbudowanych sieciach
neuronowych. Jej wady, to~brak pochodnej dla~wartości 0~oraz potencjalny problem zanikającego gradientu~--~gdy~wagi
wejść neuronu przyjmą takie wartości, że~do~funkcji aktywacji będą trafiać wartości ujemne, wówczas neuron
będzie ciągle generować wartość wyjściową równą zero. Wówczas potocznie mówi się~o~,,śmierci'' neuronu.

TODO wykres i odnośnik

\subsubsection{Funkcja softplus}
Jest to~wartiant funkcji ReLU, który jest jej ,,wygładzoną'' wersją. Posiada ciągłą pierwszą pochodną będącą funkcją
sigmoidalną.

TODO wykres i odnośnik

\subsubsection{Funkcja softmax}
Funkcja ta~najczęściej stosowana jest w~warstwie wyjściowej sieci. Bywa nazywana \textbf{znormalizowaną funkcją
eksponencjalną}, gdyż~jej wartość to~wartość funkcji eksponencjalnej podzielona przez~sumę wartości wyjściowych innych
neuronów składających się~na~daną warstwę.

TODO wykres i odnośnik

\section{Warstwowa sieć neuronowa}
Przy~tworzeniu sztucznej sieci neuronowej ważnym czynnikiem mającym istotny wpływ na~sposób rozwiązania
zadanego problemu, jest dobór odpowiedniego typu sieci. Obecnie najczęściej wykorzystywane są sieci:
\begin{itemize}
  \item jednokierunkowe,
  \item rekurencyjne,
  \item komórkowe.
\end{itemize}

\subsection{Sieć jednokierunkowa}
Jednym z~najczęściej wykorzystywanych typów sieci jest sieć jednokierunkowa. Charakterystyczną cechą takiej
sieci jest brak sprzężeń zwrotnych, tzn. sygnały przesyłane są~od~warstwy wejściowej poprzez warstwy ukryte
aż do~warstwy wyjściowej. Model przykładowej sieci warstwowej przedstawiono poniżej.

\begin{Figure}
	\centering
	\includegraphics[width=0.9\linewidth]{img/mgr_backprop_net.png}
\end{Figure}

\subsection{Wsteczna propagacja błędów} \label{ssec:backpropagation}
Wsteczna propagacja błędów jest podstawową metodą uczenia nadzorowanego wielowarstwowych jednokierunkowych
sieci neuronowych. Poprzez uczenie nadzorowane należy rozumieć proces uczenia, w~którym sieć neuronowa
otrzymuje dane wraz z~ich etykietami (spodziewanym wyjściem sieci).

W~pierwszym kroku uczenia należy zdefiniować funkcję straty $Loss(w)$ przyjmującą jako~argument wektor wag
sieci neuronowej. Funkcją tą może być średni błąd kwadratowy, dla danych ciągłych:\\
$$Loss(w)=\frac{1}{2}\sum\limits_{m}(\sigma(w^{T}x^{(m)}) - y^{(m)})^2$$
lub entropia krzyżowa, dla danych binarnych:
$$Loss(w)=-\sum\limits_m(\sigma(w^{T}x^{(m)})\log{y^{(m)}} + (1-\sigma(w^{T}x^{(m)})\log{(1-y^{(m)})})
$$
gdzie:
\begin{itemize}
  \item w - wagi sieci neuronowej,
  \item m - numer próbki uczącej,
  \item x - wartości wejściowe,
  \item y - spodziewane wartości wyjściowe,
  \item $\sigma$ - funkcja aktywacji.
\end{itemize}

Celem uczenia sieci jest minimalizacja funkcji $Loss(w)$. Na~początku należy określić gradient funkcji straty
względem wag sieci (gradient jest taki sam dla obu wymienionych powyżej funkcji).
Następnie można przystąpić do~optymalizacji tej~funkcji wykorzystując metodę gradientu prostego.
Kroki tego algorytmu przedstawiono poniżej:
\begin{enumerate}
  \item zainicjuj losowo wektor wag w,
  \item oblicz gradient funkcji straty,
  \item $w:=w-\alpha \nabla Loss(w)$, $\alpha$ - współczynnik określający długość kroku,
  \item wróć do kroku 2. jeśli nie~został spełniony warunek stopu.
\end{enumerate}

Zakładając, że~wartość błędu dla~danego przykładu trenującego wyraża się~wzorem
$$Error^{(m)}=\sigma(w^{T}x^{(m)})-y^{(m)}$$
gradient funkcji straty ma~wzór:
$$\nabla_{w}Loss=\sum\limits_{m}Error^{(m)}\sigma'(w^{T}x^{(m)})x^{(m)}$$
Ponieważ $\sigma(x)=\frac{1}{1+e^{-x}}$, pochodna funkcji sigmoidalnej wyraża się~wzorem:
$$\sigma'(x)=\sigma(x)(1-\sigma(x))$$

Dla uproszczenia załóżmy, że~w~sieci występuje tylko jedna warstwa ukryta. Wówczas gradient funkcji straty
względem wag połączeń pomiędzy warstwą wyjściową, a ukrytą:
$$\frac{\partial Loss}{\partial w_{jk}}=\frac{\partial Loss}{\partial in_k}\frac{\partial
in_k}{\partial w_{jk}}= \delta_k \frac{\partial (\sum\limits_j w_{jk}h_j)}{\partial w_{jk}} = \delta_k h_j$$
Poprzez $in_k$ oznaczono wektor wejść trafiających do~funkcji aktywacji poszczególnych neuronów warstwy
wyjściowej.

Gradient funkcjii straty względem wag połączeń pomiędzy warstwą wejściową a ukrytą: 
$$ \frac{\partial Loss}{\partial w_{ij}}=\frac{\partial Loss}{\partial in_j}\frac{\partial in_j}{\partial
w_{ij}} = \delta_j \frac{\partial (\sum\limits_j w_{ij}x_i)}{\partial w_{ij}} = \delta_j x_i $$

$$ \delta_k = \frac{\partial}{\partial in_k}(\sum\limits_k \frac{1}{2}
[\sigma(in_k)-y_k]^2)=[\sigma(\in_k)-y_k]\sigma'(in_k)$$

$$ \delta_j = \sum\limits_k\frac{\partial Loss}{\partial in_k}\frac{\partial in_k}{\partial
in_j}=\sum\limits_k \delta_k \cdot \frac{\partial}{\partial in_j}(\sum\limits_j w_{jk}\sigma(in_j))=[\sum\limits_k \delta_k
w_{jk}]\sigma'(in_j)$$

Jak widać w~ostatnim wzorze: wykorzystujemy gradient obliczony dla~warstwy wyjściowej
do~obliczenia gradientu dla~poprzedniej warstwy. Przy~większej liczbie warstw w~sieci gradienty
dla~kolejnych warstw są~obliczane analogicznie (na~podstawie wartości obliczanych w~następnych warstwach).
Stąd właśnie metoda ta nosi nazwę wstecznej propagacji błędów.


\section{Ogólna koncepcja głębokiego uczenia}
Sama koncepcja głębokiego uczenia nie jest szczególnie nowa. Sieci
wielowarstwowe uczone za pomocą wstecznej propagacji błędów istnieją
już~od~ponad 40 lat. Sieciom tym~towarzyszy jednak poważna wada: przy~wielu
warstwach w~sieci, wagi wyjść początkowych warstw są aktualizowane w~bardzo
nieznacznym stopniu (na ogół: im~większa odległość warstwy od~wyjścia,
tym~mniejsze korekcje wag). W literaturze problem ten nazywany jest problemem
zanikającego gradientu (\textit{ang. vanishing gradient problem}). Z~tego
powodu do~efektywnego uczenia metodą wstecznej propagacji błędów konieczne jest
stosowanie niewielkiej liczby warstw (w praktyce stosowano przeważnie
do~czterech warstw). Choć teoretycznie zastosowanie dwóch warstw
wystarcza do aproksymacji dowolnej funkcji, to~jednak może wymagać bardzo dużej
liczby neuronów w~warstwie ukrytej (w~niektórych problemach liczba neuronów w~warstwie ukrytej może rosnąć
wykładniczo względem rozmiaru danych wejściowych).

Rozwiązaniem problemu zanikającego gradientu jest zastosowanie dodatkowego
etapu: tzw.~uczenia wstępnego (\textit{ang.~pre-training}). W~tym~etapie
optymalizowaną funkcją nie~jest funkcja błędu (jak~w~przypadku metody wstecznej
propagacji błędu), ale~funkcja prawdopodobieństwa wystąpienia danych p(v),
gdzie v stanowi dane wejściowe sieci. Celem, do~którego dąży etap uczenia
wstępnego jest takie uformowanie funkcji p(v), by dla danych podobnych do tych obserwowanych w~zbiorze
uczącym wartość funkcji prawdopodobieństwa była jak największa, natomiast dla pozostałych danych:
jak~najmniejsza.

Opisana metoda uczenia wstępnego zakłada, że~po~kolei uczona jest każda
z~warstw (nie wszystkie na raz), zaczynając od~warstwy wejściowej, kończąc
na~wyjściowej. W~jednej iteracji algorytmu uczone są wagi tylko jednej z~warstw.
W~pierwszym kroku algorytmu uczona jest pierwsza warstwa ukryta i~efektem
uczenia powinno być wyodrębnienie takich cech danych, które odróżniają
je~od~danych losowych. W~drugim etapie kolejna warstwa jest uczona w~taki
sposób, by~była w~stanie odróżnić kombinację cech występujących w~danych
od~losowej kombinacji cech itd. Istotną różnicą tej metody w~stosunku do~metody
wstecznej propagacji błędu jest uczenie tylko jednej warstwy na~raz
(po~nauczeniu danej warstwy jej wagi nie~są już modyfikowane w~etapie uczenia
wstępnego). Tak nauczona sieć, choć jeszcze nie~nadaje się~do~przeprowadzania
klasyfikacji, to~dobrze modeluje charakter danych, tzn.~właściwości danych,
które często pojawiają się~w~zestawie uczącym.

\begin{figure}[H]
	\centering
	\includegraphics[width=0.2\linewidth]{img/hierarchical-learning_cropped.jpg}
	\caption{cs.stanford.edu}
\end{figure}

Po~etapie uczenia wstępnego następuje tzw. dostrajanie sieci
(\textit{ang.~fine-tuning}). W~tym etapie dodawana jest warstwa wyjściowa sieci.
Następnie cała sieć jest uczona danymi etykietowanymi (uczenie nadzorowane)
metodą wstecznej propagacji błędu. Dzięki zastosowaniu uczenia wstępnego
sztuczna sieć neuronowa znacznie lepiej generalizuje funkcję, którą ma~modelować
(tzn.~jest znacznie mniej podatna na~tzw.~overfitting). Ponadto, do~dostrajania
jest potrzebna znacznie mniejsza ilość danych niż do~pierwszego etapu. Zatem
do~uczenia sieci wystarczy, by~tylko niewielka część danych uczących była
etykietowana.

\section{Modelowanie funkcji prawdopodobieństwa}
Cechą wspólną wszystkich mechanizmów głębokiego uczenia badanych w~ramach tej~pracy magisterskiej jest
wykorzystanie etapu uczenia wstępnego, którego celem jest odpowiednie zamodelowanie
funkcji rozkładu prawdopodobieństwa, tzn.~osiągnięcie takiego rozkładu prawdopodobieństwa p(v),
że~danym~wejściowym v, które są podobne do~danych uczących, będą odpowiadały wysokie wartości
funkcji prawdopodobieństwa, natomiast pozostałym danym (w~szczególności danym losowym) będzie odpowiadało
niskie prawdopodobieństwo.

Do modelowania funkcji prawdopodobieństwa można wykorzystać wiele mechanizmów. W~tej~pracy omówiona zostanie
jedynie \textbf{Ograniczona Maszyna Boltzmanna}(\textit{ang. Restricted Boltzmann
Machine}).

\subsection{Autoenkoder}
Autoenkoder \cite{Autoencoder} to~sieć neuronowa, która składa się~z~trzech warstw:
\begin{itemize}
	\item warstwy wejściowej,
	\item warstwy ukrytej,
	\item warstwy wyjściowej.
\end{itemize}

\begin{figure}[H]
	\centering
	\includegraphics[width=0.5\linewidth]{img/autoencoder.png}
	\caption{autoenkoder}
\end{figure}

Dodatkowo liczba neuronów w~warstwie wyjściowej jest równa liczbie neuronów w~warstwie wejściowej.
Celem uczenia takiej sieci jest osiągnięcie stanu, w~którym wartości na~wyjściu sieci są równe
wartościom wejściowym tej sieci. Wówczas sieć ma za zadanie wykształcenie przekształcenia tożsamościowego.

\subsubsection{Zastosowania}
Zazwyczaj w~warstwie ukrytej takiej sieci umieszcza się~mniej neuronów niż~w~dwóch pozostałych warstwach.
Neurony tej warstwy w~procesie uczenia zaczynają wykrywać różne często pojawiające się powiązania
wśród wartości wejściowych (np.~mogą zauważyć, że gdy pierwsza z~wartości jest równa jeden,
wówczas trzecia i czwarta są równe zero itp.).
Pozwala to~na~kompresję danych.

Innym zastosowaniem mechanizmu autoenkodera jest korekcja danych. Wówczas w~warstwie ukrytej
może być więcej neuronów niż w~warstwie wejściowej/wyjściowej. Tak nauczona sieć, po~otrzymaniu na~wejściu
wektora wejściowego zawierającego pewne błędy w~danych (np.~szumy), potrafi dokonać korekty danych,
a dane poprawione pojawią się~na~wyjściu sieci.

W~ogólności warstwa wejściową oraz warstwa ukryta autoenkodera tworzą razem mechanizm kodujący, a~warstwa ukryta wraz
z~warstwą wyjściową tworzą mechanizm dekodujący.

\subsubsection{Funkcja kosztu}
Dla danych ciągłych ($x\in\mathbb{R}$) jako funkcję kosztu typowo stosuję się średni kwadrat błędu:
\begin{equation*}
\frac{1}{2N}\sum\limits_{i=1}^{s_1}(x-\hat{x})^2
\end{equation*}
gdzie $N$ to liczba klasyfikowanych przykładów, $s_1$ to długość wektora wejściowego (jak~również wyjściowego),
$x$ to wektor wartości wejściowych, a $\hat{x}$ to wektor wartości wyjściowych. Suma dzielona jest przez $2N$ zamiast
$N$ w~celu~wygodniejszego liczenia pochodnej w~procesie uczenia. Zabieg ten ma charakter czysto estetyczny.

\subsubsection{Regularyzacja rozrzutu}
W~autoenkoderze staramy się~zapewnić, by~średnia wartość na wyjściu każdego neuronu warstwy ukrytej
(średnia liczona po~przykładach trenujących) była bliska pewnej z~góry przyjętej wartości zwanej \textbf{parametrem
rozrzutu} (\textit{ang.~sparsity parameter})). Zwykle parametr ten przyjmuje niewielkie wartości (np. 0,05). Dzięki temu
dla~większości danych wejściowych niewiele neuronów warstwy ukrytej będzie ,,aktywowanych'', a~więc każdy neuron będzie
rozpoznawał jedną wybraną zależność pomiędzy danymi i~będzie stwierdzał czy~w~danym wektorze wejściowym jest ona obecna
(wartość na~wyjściu neuronu równa 1) czy też nie (wartość na wyjściu neuronu równa 0). W~celu osiągnięcia odpowiedniego
rozrzutu funkcja celu musi przyjmować odpowienio wyższe wartości, gdy wartość średnia dla wzbudenia neuronu warstwy
ukrytej jest różna od~zadanego parametru rozrzutu. Założony cel można osiągnąć poprzez zastosowanie funkcji kosztu
w~postaci dyvergencji Kullbacka-Leiblera:
\begin{equation*}
\sum\limits_{j=1}^{s_2}KL(\rho||\hat{\rho_j}) = \sum\limits_{j=1}^{s_2}\rho \log\frac{\rho}{\hat{\rho_j}} +
(1-\rho)\log\frac{1-\rho}{1-\hat{\rho_j}}
\end{equation*}
gdzie $s_2$ to liczba neuronów w~warstwie ukrytej, j to~numer neuronu w~warstwie ukrytej, $\rho$ to parametr rozrzutu,
a~$\hat{\rho_j}$ to~średnia wartość na~wyjściu j-tego neuronu wartstwy ukrytej.

Po~uwzględnieniu regularyzacji, funkcja kosztu przyjmie postać:
\begin{equation*}
J(W,b)=\frac{1}{N}\sum\limits_{i=1}^{s_1}(x-\hat{x})^2 + \beta\sum\limits_{j=1}^{s_2}KL(\rho||\hat{\rho_j})
\end{equation*}
gdzie $W$ i $b$, to odpowiednio wagi sieci i~bias.

\subsubsection{Uczenie}
Do~uczenia sieci wykorzystywana jest metoda wstecznej propagacji błędów (opisana we~wcześniejszej części pracy),
stąd proces zmiany wag w~sieci zaczyna się~od~warstwy wyjściowej. Zdefiniujmy różnicę pomiędzy wartościami wyjściowymi
a wejściowymi jako $\delta_i^{(3)}$, gdzie $i$ jest numerem neuronu w~warstwie wyjściowej.
Wagi warstwy wyjściowej aktualizujemy wg wzoru \ref{eqn:wagi_autenc}:
\begin{equation}
    \begin{split}
    W'_i &= W_i + \alpha\delta^{(2)}_i \\
    \delta_i^{(2)} &= \left( \left( \sum\limits_{j=1}^{s_{2}} W^{(2)}_{ji} \delta^{(3)}_j \right)
    + \beta \left( - \frac{\rho}{\hat\rho_i} + \frac{1-\rho}{1-\hat\rho_i} \right) \right) f'(z^{(2)}_i)
    \end{split}
    \label{eqn:wagi_autenc}
\end{equation}
gdzie $f'$ to pochodna funkcji aktywacji używanej w~neuronach, a~$z^{(2)}_i$, to~wartości preaktywacji neuronów warstwy
ukrytej (wartości, które trafiają do~funkcji aktywacji tych neuronów).

Wagi wastwy ukrytej odpowiadają transponowanej macierzy wag warstwy wyjściowej. Jest tak~dlatego, że~jak~wcześniej
wspomniano, warstwa wyjściowa odpowiada za~operację odwrotną do~operacji wykonywanej przez~warstwę ukrytą (odpowiednio:
dekodowanie i kodowanie).

\subsection{Ograniczona Maszyna Boltzmanna}
Ograniczona Maszyna Boltzmanna jest sztuczną siecią neuronową składającą się~z~dwóch warstw neuronów:
\begin{itemize}
  \item warstwy neuronów wejściowych,
  \item warstwy neuronow ukrytych.
\end{itemize}

\begin{figure}
	\centering
	\includegraphics[width=0.5\linewidth]{img/RBM.png}
	\caption{Ograniczona Maszyna Boltzmanna}
\end{figure}

Cechą charakterystyczną tego mechanizmu jest jego umiejętność do~odszukiwania cech danych wejściowych
(\textit{ang. feature extraction}). Przykładowo Ograniczona Maszyna Boltzmanna, którą uczono obrazkami, będzie
wykrywać charakterystyczne łuki czy krawędzie.

Dzięki tej właściwości mechanizm może być wykorzystywany m.in. do~korekcji danych czy~wręcz do~ich~generowania
(np.~do~generowania pisma ręcznego).

\subsubsection{Prawdopodobieństwo stanów sieci}
W~Ograniczonej Maszynie Boltzmanna każdy stan sieci (tzn. zestaw wartości wyjść poszczególnych neuronów)
ma~określone prawdopodobieństwo zadane wzorem:

$p(v,h)=\frac{1}{Z}e^{-E(v,h)}$, gdzie:
\begin{itemize}
  \item v to macierz wyjść neuronów wejściowych,
  \item h to macierz wyjść neuronów ukrytych,
  \item Z to czynnik normalizujący (tzw.~funkcja partycji),
  \item $E(v,h)=-a^{T}v-b^{T}h-v^{T}Wh$,
  \item a,b oraz W to macierze wag sieci.
\end{itemize}

\subsubsection{Inferencja}
Ze~względu na~to, że~neurony nie~łączą się~ze~sobą w~obrębie tej~samej warstwy, wartość na~wyjściu każdego
neuronu w~danej warstwie można uzyskać wyłącznie na~podstawie wartości wyjść neuronów drugiej warstwy.
Prawdopodobieństwa wzbudzeń neuronów można efektywnie obliczyć korzystając z~wzorów:
\begin{equation}
    \begin{split}
        p(h_{j}=1|v)&=\sigma(b_{j}+\sum\limits_{i=1}^{m}w_{i,j}v_{i}) \\
        p(v_{i}=1|h)&=\sigma(a_{i}+\sum\limits_{j=1}^{n}w_{i,j}h_{j})
    \end{split}
	\label{eqn:rbm_inference}
\end{equation}
gdzie: $\sigma$ to funkcja aktywacji, $i$ oraz $j$ to numery kolejnych neuronów (odpowiednio: wejściowych i~ukrytych).


\subsubsection{Uczenie}
Celem uczenia sieci jest znalezienie takich parametrów a,b i W, że~wartość funkcji prawdopodobieństwa p(v)
dla~danych obserwowanych będzie wysoka, a~dla~pozostałych danych niska. W~praktyce zamiast maksymalizować
funkcję prawdopodobieństwa p(v), minimalizuje się funkcję średniego ujemnego log-prawdopodobieństwa:
$$\frac{1}{T}\sum\limits_{t}-\log{p(v^{(t)})}$$

Do~minimalizacji funkcji metodami gradientowymi konieczne jest obliczenie pochodnej po~parametrach sieci
dla tej funkcji:
$$\frac{\partial-\log{p(v^{(t)})}}{\partial{\theta}}=\mathbb{E}_h[\frac{\partial{E(v^{(t)},h)}}{\partial{\theta}}\mid{v^{(t)}}]-\mathbb{E}_{v,h}[\frac{\partial{E(v,h)}}{\partial{\theta}}]$$

Pierwszy ze~składników sumy występującej w~powyższym wzorze to~tzw. składnik pozytywny. Drugi składnik
to~tzw.~składnik negatywny. Pierwszy z~nich odpowiada zwiększaniu wartości funkcji prawdopodobieństwa
dla~obserwowanych danych, a drugi zmniejsza prawdopodobieństwo danych nieobserwowanych. Drugi składnik jest
skomplikowany obliczeniowo, gdyż występuje w~nim iteracja po~wszystkich możliwych stanach sieci, których
liczba rośnie wykładniczo względem liczby neuronów w~sieci. Z~tego powodu dokonuje się estymacji. Jedną
z~wykorzystywanych metod jest tzw. próbkowanie Gibbsa. 

\textbf{Próbkowanie Gibbsa} pozwala wyeliminować problem obliczania wartości oczekiwanej dla~wszystkich
możliwych stanów sieci. Zamiast tego wartość ta~jest estymowana za~pomocą algorytmu, którego kroki
przedstawiono poniżej.

Próbkowanie Gibbsa (algorytm):
\begin{enumerate}
  \item Oblicz wartości wyjściowe dla~każdego neuronu z~warstwy ukrytej na~podstawie danych wejściowych
  (korzystając z~wzorów \ref{eqn:rbm_inference}).
  \item Oblicz wartości wyjściowe dla~każdego neuronu z~warstwy wejściowej na~podstawie obliczonych wartości
  wyjściowych neuronów ukrytych (korzystając z~wzorów \ref{eqn:rbm_inference}).
\end{enumerate}

Kroki 1 i 2 mogą być powtarzane wielokrotnie, choć w~praktyce okazuje się, że wystarczy jeden
przebieg algorytmu. Otrzymane w~punkcie drugim ,,odtworzone'' wartości wejścia oznaczane są~jako $\tilde{v}$.

\paragraph{Aktualizacja parametrów sieci}
Parametry (wagi) sieci aktualizowane są według następujących wzorów:
\begin{itemize}
  \item $W_t=W_{t-1}+\alpha(h(v^{(t)})v^{(t)T}-h(\tilde{v})\tilde{v}^{T})$
  \item $b_t=b_{t-1}+\alpha(h(v^{(t)})-h(\tilde{v}))$
  \item $a_t=a_{t-1}+\alpha(v^{(t)})-\tilde{v})$
\end{itemize}
\vspace{1cm}

Przez $h(v)$ oznaczono wartości wyjściowe neuronów ukrytych obliczone na~podstawie wartości wyjściowych
neuronów wejściowych.

\chapter{Splotowe sieci neuronowe}
Splotowe sieci neuronowe są wielowarstwowymi sieciami jednokierunkowymi. Charakteryzuje je~możliwość
wykrywania wzorców występujących w~różnych fragmentach przetwarzanego obrazu (np. rozpoznawanie oka
w~dowolnym fragmencie obrazka). Zasada ich~działania była inspirowana neurobiologią, a~mianowicie analizowano
sposób przetwarzania informacji przez~ośrodek wzrokowy kota.

\section{Filtry splotowe}
Splot dyskrentny jako pojęcie matematyczne jest zdefiniowany w~następujący sposób:
$$ f\ast g[n]\defeq \sum\limits_{m=-\infty}^{\infty}f[m]g[n-m] = \sum\limits_{m=-\infty}^{\infty}f[n-m]g[m]$$
gdzie $f$ i $g$ to~ciągi splatane, a~zapis $f[n]$ to~zapis oznaczający $n$-ty wyraz ciągu $f$.

W~cyfrowym przetwarzaniu obrazów filtry splotowe znajdują bardzo szerokie zastosowanie, gdyż w~zależności
od~dobranej maski filtra, osiągane są różne rezultaty.

\subsection{Przykładowe filtry splotowe}
\subsubsection{Filtr Gaussa}
By~lepiej zrozumieć działanie filtrów splotowych warto posłużyć się~przykładami. Jednym z~najczęściej stosowanych
mechanizmów tego typu jest filtr Gaussa, którego zadaniem jest wygładzenie (inaczej: rozmazanie) obrazu.
Maska tego filtru została przedstawiona na~rysunku \ref{rys:maska-gauss}.
\begin{figure}[H]
	\centering
	\includegraphics[width=0.25\linewidth]{img/gauss-conv-kernel.png}
	\caption{Maska filtru Gaussa}
	\label{rys:maska-gauss}
\end{figure}

Dla~obrazu źródłowego (patrz rys.~\ref{rys:img-src-gauss}), aby~otrzymać obraz przekształcony
(rys.~\ref{rys:output-gauss}) należy policzyć średnią ważoną dla~otoczenia każdego przekształcanego piksela, a~więc:
\begin{enumerate}
    \item Przemnożyć wartość każdego piksela z~odpowiadającą mu wartością maski.
    \item Dodać wszystkie iloczyny do~siebie.
    \item Znormalizować uzyskaną wartość poprzez podzielenie jej przez~sumę wag tworzących maskę (krok wykonywany tylko
    przy~wykorzystaniu niektórych filtrów).
\end{enumerate}

Tak~uzyskana wartość to~w~przekształconym obrazie nowa wartość piksela. Operacja splotu jest zwykle wykonywana
dla~każdego piksela oryginalnego obrazu.

\begin{figure}[H]
	\centering
	\includegraphics[width=0.5\linewidth]{img/gauss-conv-src-img.png}
	\caption{Przykładowy obraz, który może zostać przekształcony przy~użyciu filtru Gaussa. Poszczególne wartości
	w~komórkach odpowiadają wartościom pikseli w~obrazie niekolorowym (odcienie szarości)}
	\label{rys:img-src-gauss}
\end{figure}

\begin{figure}[H]
	\centering
	\includegraphics[width=0.5\linewidth]{img/gauss-conv-output.png}
	\caption{Obraz zawierający piksel przekształcony przy~użyciu filtru Gaussa}
	\label{rys:output-gauss}
\end{figure}

Efekt działania filtru Gaussa na~prawdziwym obrazie przedstawiono na~rysunku \ref{rys:gauss-conv-example}.
\begin{figure}[H]
	\centering
	\includegraphics[width=0.4\linewidth]{img/gauss-conv-real-example.jpg}
	\caption{Efekt działania filtru Gaussa na~prawdziwym obrazie. Na~górze: obraz oryginalny, na~dole: obraz
	przekształcony}
	\label{rys:gauss-conv-example}
\end{figure}

Jak~można zaobserwować, w~filtrze Gaussa na~nową wartość piksela mają częściowo wpływ piksele sąsiednie
i~przede~wszystkim stara wartość piksela.

\subsubsection{Filtr Laplace'a}
Zadaniem tego filtru jest wyostrzanie obrazu. Efekt ten jest osiągany poprzez usunięcie wpływu sąsiednich pikseli
na~wartość każdego przekształcanego piksela. Jest to~poniekąd filtr o~działaniu odwrotnym do~filtru Gaussa. Jądro splotu
dla~omawianego mechanizmu przedstawiono na~rysunku \ref{rys:maska-laplace}.

\begin{figure}[H]
	\centering
	\includegraphics[width=0.25\linewidth]{img/laplace-conv-kernel.png}
	\caption{Maska filtru Laplace'a}
	\label{rys:maska-laplace}
\end{figure}

Działanie filtru Laplace'a zaprezentowano na~rysunku \ref{rys:laplace-conv-example.jpg}.

\begin{figure}[H]
	\centering
	\includegraphics[width=\linewidth]{img/laplace-conv-example.png}
	\caption{Efekt działania filtru Laplace'a na~prawdziwym obrazie. Po~lewej: obraz oryginalny, po~prawej: obraz
	przekształcony}
	\label{rys:laplace-conv-example}
\end{figure}

Maski dla~zaprezentowanych filtrów (tj.~dla~filtru Gaussa i~filtru Laplace'a) są~z~góry ustalone i~nie~ulegają zmianie
w~czasie działania algorytmu.

Splotowa sieć neuronowa zamiast wykorzystywać z~góry określone maski, ,,uczy się ich'',
a~więc stara się~tak dobrać ich~wagi, aby~jak~najlepiej rozpoznawać wyznaczone obiekty na~obrazach. Poszczególnym
wartościom w~masce filtra odpowiadają wagi splotowej sieci neuronowej.

\subsection{Dopełnienie (\textit{ang.~padding})}
Przy~stosowaniu filtrów splotowych pojawia się problem: jaką wartość powinny otrzymać piksele obrazka
znajdujące się na~jego krawędzi. Jest on~rozwiązywany poprzez rozszerzanie obrazka o~dodatkowe piksele
na~jego krawędziach. Można tego dokonać m.in. poprzez:
\begin{itemize}
  \item dopełnienie obrazka czarnymi pikselami (\textit{ang.~zero-padding}),
  \item dopełnienie obrazka odbiciami lustrzanymi pikseli przy krawędziach.
\end{itemize}

\section{Inferencja} \label{sec:inferencja}
W~trakcie działania sieci na~obrazie dokonywanie jest filtrowanie splotowe (przy wykorzystaniu różnych
filtrów). Po~takim filtrowaniu otrzymywane jest $n\cdot m$ obrazów, nazywanych mapami cech
(\textit{ang.~feature maps}), gdzie n~-~liczba początkowych obrazów (tzw.~kanałów wejściowych), m~-~liczba
użytych filtrów.

Po~dokonaniu filtrowania splotowego obrazy są skalowane na~mniejsze (tzw.~faza pooling/subsampling) i~znów
stosowane są~filtry splotowe. Oba kroki (filtrowanie i~skalowanie) są~powtarzane wielokrotnie
(zależy od~liczby warstw sieci), aż~do~osiągnięcia odpowiednio wyskokopoziomowych cech. Liczba warstw oraz
liczba filtrów splotowych wykorzystanych w~każdej z~nich jest ustalana podczas fazy projektowania sieci.

\begin{figure}[H]
	\centering
	\includegraphics[width=\linewidth]{img/convnet.png}
\end{figure}

Ostatnią warstwą sieci (zazwyczaj otrzymującą dużo bardzo małych obrazów) jest warstwa zawierająca neurony
sigmoidalne. Każdy neuron warstwy wyjściowej jest połączony ze~wszystkimi wartościami otrzymanymi na~wyjściu
poprzedniej warstwy.

\subsection{Skalowanie}
Po zastosowaniu n różnych filtrów splotowych w~danej warstwie na~m~różnych mapach cech, powstaje $n\cdot m$
kolejnych map cech. Stąd ilość przetwarzanych danych szybko rośnie wraz z~dokładaniem kolejnych warstw
w~sieci. Aby temu przeciwdziałać stosuje się skalowanie obrazów pomiędzy warstwami dokonującymi splotu.
Najczęściej obraz skalowany jest poprzez:
\begin{enumerate}
  \item podzielenie obrazka na~nienachodzące na~siebie kwadratowe obszary,
  \item wybranie z~każdego obszaru piksela o~największej wartości (tzw.~\textit{max-pooling}). 
\end{enumerate}

Alternatywnie, w~drugim kroku algorytmu można wybrać medianę wartości pikseli\\
(\textit{ang.~mean-pooling}) lub~wartość średnią (\textit{average-pooling}).

\subsection{Normalizacja}
W~sieciach splotowych często stosowanym zabiegiem, mającym na~celu zapewnienie lepszej jakości klasyfikacji jak również
szybszego uczenia, jest normalizacja danych. Operacja normalizacji może być interpretowana jako kolejna warstwa
umieszczana w~sieci na~tej~samej zasadzie, co~warstwa dokonująca splotu czy~warstwa skalująca.

\subsubsection{Lokalna normalizacja odpowiedzi} \label{sssec:normalizacja_odpowiedzi}
\textbf{Lokalna normalizacja odpowiedzi} (\textit{ang.~local response normalization}, \cite{HOG}) polega na~zapewnieniu,
że~wartości pikseli o~podobnych współrzędnych, jednak należące do~różnych map cech, należą do~rozkładu normalnego
o~odchyleniu standardowym równym 1 i~średniej równej 0. Normalizacja polega więc na~odjęciu od~wszystkich pikseli
średniej (liczonej wzdłuż różnych map cech) i~podzieleniu ich przez~odchylenie standardowe (również liczone wzdłuż
rożnych map cech).

Choć wartości odchylenia standardowego i~średniej liczone są z~uwzględnieniem wszystkich map cech, to~nie~muszą
się ograniczać do~jednego piksela. Zamiast tego mogą dotyczyć pewnego jego sąsiedztwa. Metoda ta~odpowiada
mechanizmowi obecnemu w~ośrodkach wzrokowych ssaków, u~których również dla pewnego sąsiedztwa punktu widzianego
obrazu kontrast jest normalizowany. Przykładem tego efektu jest sytuacja, w~której obiekt w~zależności od~tego,
w~jakim otoczeniu się~znajduje, wydaje się~mieć inny kolor \ref{img:chess-illusion}.

\begin{figure}[H]
	\centering
	\includegraphics[width=0.8\linewidth]{img/chess-illusion.png}
	\caption{Pola A i B mają ten sam kolor}
	\label{img:chess-illusion}
\end{figure}

\subsubsection{Normalizacja wsadowa}
Jedną z~często stosowanych obecnie metod normalizacji jest tzw.~normalizacja wsadowa (\textit{ang.~batch normalization}).
Jest stosunkowo nową metodą, opublikowaną w~2015 roku (\cite{batch-norm}). Stosuje się~ją jako
dodatkową warstwę umieszczaną za~każdą z~warstw w~pełni połączonych. Warstwa BatchNorm dokonuje normalizacji danych
w~ramach całego wsadu danych (tzw.~batch). Odejmuje~ona od~każdej wartości wejściowej średnią (liczoną dla~przykładów
w~obrębie całego wsadu, dla~każdego z~wejść z~osobna), a~następnie dzieli uzyskane wartości przez~odchylenie standardowe
(liczone analogicznie jak~średnia).
\begin{equation*}
 \widehat{x} = \frac{x - E[x]}{\sqrt{Var(x)}}
\end{equation*}

Tak~znormalizowane dane są~następnie poddawane przekształceniu liniowemu:
\begin{equation*}
y = \gamma\widehat{x} + \beta
\end{equation*}
$\gamma$ oraz $\beta$ są~parametrami, których wartość ustalana jest w~procesie uczenia. Warto zwrócić uwagę, że~jeśli
$\beta=E[x]$ i $\gamma=\sqrt{Var(x)}$ warstwa BatchNorm będzie dokonywała przekształcenia identycznościowego.

Dla~każdej z~operacji wykonywanych przez~tę~warstwę normalizującą można policzyć gradient funkcji kosztu względem
parametrów. Dzięki temu sieć z~warstwami BatchNorm może być uczona metodami gradientowymi, podobnie jak~większość
standardowych sieci.

\subsection{Regularyzacja sieci}
W~przypadku rozpoznawania danych, w~których występować może wiele różnych wzorców, konieczne jest umieszczenie w~sieci
neuronowej dużej liczby neuronów. Jednakże może to~przynieść nieporządany efekt, przy~którym sieć zbytnio dopasuje
się~do~danych (tzw.~overfitting). Sposobem na~ograniczanie tego zjawiska jest \textbf{regularyzacja} sieci.

\subsubsection{Regularyzacja L2} \label{sssec:reg_L2}
Jest~to~prawdopodobnie naczęściej używana forma regularyzacji. By~ją~zastosować, do~funkcji kosztu sieci dodawany
jest człon $\frac{1}{2}\lambda w^2$, a~gradient tego członu to~$\lambda w$. W~efekcie przy~aktualizacji wag wszystkie
z~nich są zmniejszane w~liniowy sposób:
\begin{equation*}
W := W*(1 -\lambda)
\end{equation*}
$\lambda$~to~tzw.~współczynnik regularyzacji, który im~jest większy, tym~bardziej sieć dąży do~tego,
by~mieć równe wartości wszystkich wag (\cite{L2-regularization}).

\subsubsection{Dropout}
Jest to~niezwykle efektywna, a~zarazem prosta technika pozwalająca na~zapobieganie zbytniemu dopasowaniu sieci
do~danych przy~jednoczesnym zapobieganiu \textbf{koadaptacji} neuronów, czyli~sytuacji, w~której różne neurony
przyjmują podobne wagi, przez co~nie~są od~siebie niezależne. Dropout polega na~losowym usuwaniu neuronów z~sieci
podczas procesu uczenia. Dzięki temu w~każdej iteracji uczony jest jedynie pewien podzbiór sieci (\cite{dropout}).

\begin{figure}[H]
	\centering
	\includegraphics[width=\linewidth]{img/dropout.jpeg}
	\caption{Dropout. Po~lewej: pełna sieć. Po~prawej: sieć po~dokonaniu operacji usuwania losowych neuronów.
	         Źródło: \cite{dropout}}
\end{figure}

\section{Uczenie metodą wstecznej propagacji}
Typową metodą uczenia sieci jest wsteczna propagacja błędów. W~tym celu należy obliczyć gradient funkcji
straty względem wag sieci (wartości masek filtrów splotowych) dla~warstw:
\begin{itemize}
  \item wyjściowych,
  \item skalujących,
  \item splotowych.
\end{itemize}
Obliczanie gradientu funkcji straty dla~warstw zawierających neurony sigmoidalne zostało omówione
w~sekcji (\ref{ssec:backpropagation}), stąd wyjaśnienia wymaga jedynie obliczanie gradientów dla~warstw skalujących i~splotowych.

\subsection{Obliczanie gradientu dla~warstw skalujących}
Znając gradient funkcji straty $\nabla_{y_{ijk}}l$ z~warstwy kolejnej, można w~prosty sposób policzyć gradient
w~stosunku do~wejścia warstwy dla~której jest on liczony. W~przypadku, gdy~warstwa wykorzystywała algorytm
\textit{max-pooling}, gradient względem wejść warstwy przyjme wartość:
\begin{itemize}
  \item $\nabla_{x_ijk}l = \nabla_{y_{ijk}}l$, dla~pikseli, które miały maksymalną wartość w~obszarze, 
  \item $\nabla_{x_ijk}l = 0$, dla~pozostałych pikseli.
\end{itemize}

\subsection{Obliczanie gradientu dla~warstw splotowych}
Przy~wstecznej propagacji błędów z~warstwy kolejnej otrzymujemy gradient błędu względem wyjścia
aktualnej warstwy $\nabla_{y_j}l$. Do~aktualizacji wag sieci konieczne jest policzenie gradientu funkcji
straty wzlgędem wag tej warstwy.
$$ \frac{\partial E}{\partial w_{ab}} 
= \sum\limits_{i=0}^{N-m}\sum\limits_{j=0}^{N-m}\frac{\partial E}{\partial x_{ij}^l}\frac{\partial
x_{ij}^l}{\partial w_{ab}}
= \sum\limits_{i=0}^{N-m}\sum\limits_{j=0}^{N-m}\frac{\partial E}{\partial x_{ij}^l}y^{l-1}_{(i+a)(j+b)}$$
gdzie:
\begin{itemize}
  \item $N\times N$ - rozmiar mapy cech,
  \item $m\times m$ - rozmiar filtra,
  \item $x_{ij}^l$ - wartość wejścia neuronu o~wsp. (i,j) w~warstwie l,
  \item $w_{ab}$ - wartość w~masce filtra znajdująca się~na~pozycji (a,b).
\end{itemize}

Następnie należy policzyć tzw.~delty ($\frac{\partial E}{\partial x_{ij}^l}$):
$$\frac{\partial E}{\partial x_{ij}^l} = \frac{\partial E}{\partial y_{ij}^l}\frac{\partial y_{ij}^l}{\partial
x_{ij}^l} =
\frac{\partial E}{\partial y_{ij}^l}\frac{\partial}{\partial x_{ij}^l}(\sigma(x_{ij}^l))
= \frac{\partial E}{\partial y_{ij}^l}\sigma'(x_{ij}^l)
$$

Na~koniec należy policzyć gradient funkcji błędu względem wyjść warstwy poprzedniej (po~to, by~przekazać
go~do~poprzedniej warstwy):
$$ \frac{\partial E}{\partial y_{ij}^{l-1}} =
\sum\limits_{a=0}^{m-1}\sum\limits_{b=0}^{m-1} \frac{\partial E}{\partial x^l_{(i-a)(j-b)}}
\frac{\partial x^l_{(i-a)(j-b)}}{\partial y_{ij}^{l-1}} =
\sum\limits_{a=0}^{m-1}\sum\limits_{b=0}^{m-1} \frac{\partial E}{\partial x^l_{(i-a)(j-b)}}w_{ab}$$

Należy zwrócić uwagę na~dwie rzeczy: po~piersze na~to, że~podczas obliczania gradientów również wykonywany
jest splot jednak na~obrazku o~,,odwróconych'' wierszach i~kolumnach. Dodatkowo, by~móc dokonywać splotu
na~krawędziach map cech, należy zastosować dopełnienie zerami (\textit{ang.~zero-padding}).

\subsection{Uczenie wstępne}
W~celu lepszego zainicjowania wag sieci należy przeprowadzić uczenie wstępne. W~tym~etapie sieć zamiast uczyć
się rozpoznawania obiektów, ma~za~zadanie nauczyć się~charakteru danych wejściowych (ma~zauważać
charakterystyczne cechy, podobieństwa obiektów it.p.). Do~uczenia wstępnego wykorzystuje się~mechanizmy
uczenia nienadzorowanego, takie jak:
\begin{itemize}
  \item autoenkoder,
  \item RBM (Restricted Boltzmann Machine).
\end{itemize}

Uczenie wstępne przebiega w~następujący sposób:
\begin{enumerate}
  \item wyodrębnienie losowych fragmentów obrazka (tzw.~łatki),
  \item uczenie mechanizmu uczenia nienadzorowanego (np.~RBM),
  \item zainicjowanie wag uczonej warstwy wagami z~tego~mechanizmu,
  \item policzenie map cech dla~łatek,
  \item wykonanie kroków 1-3 dla~uzyskanych map~cech (uczenie wstępne kolejnej warstwy).
\end{enumerate}

\input{tex/biblioteki-programistyczne}
\chapter{Badania}
\section{Wykorzystywana miara jakości klasyfikatora}
Podstawowymi miarami stosowanymi w~ocenie jakości sieci neuronowych są:
\begin{itemize}
    \item dokładność (\textit{ang.~accuracy}),
    \item średnia precyzja (\textit{ang.~average precision}).
\end{itemize}

\paragraph{Dokładność} wskazuje jaka część przykładów ze~zbioru ewaluacyjnego została sklasyfikowana poprawnie.
\begin{equation*}
    ACC = \frac{CP}{T}
\end{equation*}
gdzie $CP$ to~liczba poprawnie sklasyfikowanych przykładów, a~$T$ to~całkowita liczba przykładów.

Jeśli wśród przykładów istnieje spora nadreprezentacja przykładów jednej klasy wówczas wskaźnik ten~może być
niemiarodajny. Przykładowo: klasyfikator mający stwierdzać płeć wśród studentów Wydziału Elektroniki i~Technik
Informacyjnych Politechniki Warszawskiej mógłby wsakzywać dla~każdej osoby płeć męską, a~jego dokładność byłaby
dość wysoka. Dlatego w~takich przypadkach badając jakość sieci należy posłużyć się~również dodatkowymi miarami.

\paragraph{Precyzja} to~miara, która liczona jest dla~każdej z~klas z~osobna i~wyraża ile~przykładów sklasyfikowanych
 jako należące do~danej~klasy faktycznie należy do~tej klasy (przykładowo: jaka część obrazów sklasyfikowanych jako
 obraz z~psem faktycznie jest obrazem przedstawiającym to~zwierzę). W~ocenie jakości sieci neuronowych rozpoznających
 więcej niż~dwie klasy stosuje się~uśrednioną precyzję, tj.~średnią z~precyzji uzyskanych dla~każdej z~klas.
 \begin{equation*}
    PPV = \frac{TP}{TP+FP}
\end{equation*}
gdzie $TP$ to~prawidłowo rozpoznane przykłady danej klasy, a~$FP$ to~przykłady niepoprawnie sklasyfikowane jako
 należące do~tej klasy.

\section{Architektura sieci neuronowej}
Podczas tworzenia splotowej sieci neuronowej, należy dobrać wiele hiperparametrów takich jak:
\begin{itemize}
    \item liczba warstw splotowych,
    \item liczba jąder stosowanych do~wykonywania splotów w~każdej z~warstw splotowych,
    \item rozkład warstw typu max-pooling (\ref{sec:inferencja}),
    \item rozkład warstw normalizujących (\ref{sssec:normalizacja_odpowiedzi}),
    \item współczynnik uczenia (\ref{ssec:backpropagation}).
\end{itemize}

Ogólne zasady dotyczące pierwszych dwóch z~wymienionych punktów zostały opisane w~artykule
,,Rethinking the Inception Architecture for Computer Vision''(\cite{RIACV}). Posiłkując się~przytoczoną pracą można
wymienić kilka wskazówek przydatnych przy~ustalaniu hiperparametrów sieci:
\begin{itemize}
    \item unikanie zbyt małej liczby neuronów w~warstwach, w~sczególności w~warstwach początkowych. Warto zastosować
          kilkukrotność/kilkunastokrotność spodziewanej liczby klas, które ma~rozpoznawać sieć
          (w~przypadku CIFAR-10 jest to~10 klas). Warstwy końcowe mogą zawierać mniejszą liczbę neuronów, niż warstwy
          poprzednie.
    \item zmniejszenie rozmiaru danych wejściowych poprzez zastosowanie metod, takich jak:
          \begin{itemize}
              \item usunięcie brzegów, gdyż~zwykle zawierają mało istotne dane,
              \item zmniejszenie rozmiaru obrazka poprzez zastosowanie sklaowania.
          \end{itemize}
    \item używanie niewielkich filtrów splotowych (np. 3x3 lub 5x5 zamiast 7x7). Lepsze efekty daje zastosowanie dwóch
          warstw splotowych o~maskach 3x3 niż jednej maski 7x7,
    \item warto zacząć od 2 do 5 warstw splotowych (tyle samo warstw skalujących i~normalizujących), następnie zwiększać
          liczbę masek używanych w~warstwach splotowych na~przemian ze~zwiększaniem liczby warstw.
\end{itemize}

\subsection{Architektura badanej sieci}
Badana sieć w~swojej podstawowej wersji bazuje na~architekturze AlexNet przedstawionej w~artykule \cite{AlexNet}.
Po dokonaniu drobnych modyfikacji w~końcowych etapach przetwarzania obrazu, sieć składa się~z~następujących warstw:
\begin{enumerate}
    \item Warstwy splotowej z~64 maskami o rozmiarze 5x5x3 (wysokość x szerokość x liczba objętych kanałów).
          Maska przesuwana jest zawsze o~1~piksel (w~kierunku pionowym i~poziomym).
    \item Warstwy skalującej typu max-pooling o~wielkości filtra 3x3x1 (wysokość x szerokość x liczba objętych kanałów).
          Filtr jest przesuwany o~2~piksele (w~kierunku pionowym i~poziomym)
    \item Warstwy normalizującej (normalizacja lokalnej odpowiedzi).
    \item Warstwy splotowej z~64 maskami o rozmiarze 5x5x64 (wysokość~x~szerokość~x~liczba objętych kanałów).
          Maska przesuwana jest zawsze o~1 piksel (niezależnie od~kierunku przesuwania maski).
    \item Warstwy normalizującej (normalizacja lokalnej odpowiedzi).
    \item Warstwy skalującej typu max-pooling o~wielkości filtra 3x3x1 (wysokość~x~szerokość~x~liczba objętych kanałów).
          Filtr jest przesuwany o~2~piksele (w~kierunku pionowym i~poziomym).
    \item Warstwy w~pełni połączonej (standardowa warstwa w~sieciach neuronowych) z~384 neuronami i~funkcją aktywacji
          typu ReLU.
    \item Warstwy w~pełni połączonej z~192 neuronami i~funkcją aktywacji
          typu ReLU.
    \item Warstwy wyjściowej (również w~pełni połączonej) z~10 neuronami (tyle samo, co~klas do~rozpoznawania).
          Warstwa wyjściowa zawiera funkcję aktywacji typu softmax.
\end{enumerate}

\subsubsection{Przetwarzanie wstępne}
Dane wejściowe przed~tym, jak~trafią do~sieci neuronowej, poddawane są~przetwarzaniu wstępnemu. Sprowadza się~ono
do~przycięcia obrazka, tak~by~jego rozdzielczość wyniosła 24x24 piksele (oryginalna: 32x32 piksele). W~procesie
uczenia obrazek przycinany jest losowo, a~w~przypadku obrazka klasyfikowanego~--~wybierany jest środkowy fragment.
Dodatkowo, jeśli obrazek ma~być wykorzystywany w~procesie uczenia, poddawany jest on~zniekształceniu.

Po~wczytaniu i~przycięciu obrazka (w~przypadku uczenia: również po~zastosowaniu zniekształcenia) każdy z~obrazów jest
normalizowany (niezależnie od~innych). Normalizacja polega na~zapewnieniu, że~subpiksele w~każdym kanale
przyciętego obrazka (czerwonym, zielonym i~niebieskim) mają średnią wartość równą zero i~odchylenie standardowe równe~1.

% TODO dodać obrazek przedstawiający graf operacji
% TODO ustawienie seeda do generatora liczb losowych w tensorflow -- tf.set_random_seed(1)
%      http://stackoverflow.com/questions/36288235/how-to-get-stable-results-with-tensorflow-setting-random-seed


\section{Plan badań}
Celem niniejszej pracy naukowej jest sprawdzenie jak istotny wpływ na~dokładność klasyfikacji mają 2~czynniki:
\begin{itemize}
    \item regularyzacja typu L2 (\ref{sssec:reg_L2}),
    \item normalizacja lokalnej odpowiedzi (\ref{sssec:normalizacja_odpowiedzi}).
\end{itemize}

W~artykule ,,Practical Recommendations for Gradient-Based Training of Deep Architectures''
(\cite{practical-gradient-based}) skrótowo omówiono problem doboru hiperparametrów sieci. Jednym ze~sposobów
przedstawionych w~opracowaniu jest określenie wartości brzegowych dla~optymalizowanych hiperparametrów,
a~następnie zbadanie przestrzeni między. Po~zbadaniu zachowania sieci dla~tej przestrzeni hiperparametrów, można podjąć
decyzję o~przyjęciu jednego z~zestawów hiperparametrów dla~sieci lub~rozszerzyć pole poszukiwań o~kolejne obszary.

Dla~parametru regularyzacji~L2 (tzw.~\textbf{weight decay}) jako~górne ograniczenie przyjęto początkowo wartość~0.05.
Wybór bazował na~tym, że~w~podobnych sieciach, tj.~przeznaczonych do~identyfikacji obiektów przedstawianych na~obrazkach
z~bazy ImageNet (\cite{imagenet}), wartość tego hiperparametru nie~przekraczała~0.03. Badanie miało sprawdzić
również jak~sieć zachowywałaby~się~bez regularyzacji wag. Stąd jako dolne ograniczenie przyjęto wartość~0.

Dla~parametru decydującego o~wpływie normalizacji lokalnej odpowiedzi (tzw.~parametr $\alpha$) jako ograniczenie górne
przyjęto początkowo wartość~0.001, a~jako ograniczenie dolne wartość~0. Usprawiedliwienie dla~tych decyzji było
takie samo, jak~dla~wyborów dokonanych przy~hiperparametrze regularyzacji~L2.

Dla~każdej pary parametrów sieć była uczona 100,000 mini-zestawów danych (tzw.~\textbf{minibatch}), z~których
każdy zawierał 128 przykładów uczących. Po~każdym kroku uczenia pojedynczym mini-zestawem danych, sprawdzano
dokładność sieci na~zbiorze testowym. Po~wykonaniu wszystkich kroków uczenia dla~danej pary hiperparametrów
brano średnią dokładność sieci na~zbiorze testowym ze~100 ostatnich kroków uczenia. Wyniki przedstawiono w~tabelce
(\ref{table:wyniki1}).

\section{Środowisko sprzętowe}
Badania zostały wykonane z wykorzystaniem następującego zestawu komputerowego:
\begin{itemize}
    \item \textbf{procesor}~--~Intel Core i7-4771 3,5Ghz (8 rdzeni),
    \item \textbf{płyta główna}~--~MSI~B85M-G43,
    \item \textbf{karta graficzna}~--~MSI GeForce GTX 780 Ti,
    \item \textbf{pamięć RAM}~--~2 x GoodRam 8GB 1600 MHz.
\end{itemize}

\section{Wyniki badań}
\begin{table}[H]
    \centering
    \begin{tabular}{|l|l|l|l|l|l|}
      \hline
                       & $\lambda$ = 0.0005 & $\lambda$ = 0.001 & $\lambda$ = 0.005 & $\lambda$ = 0.01 & $\lambda$ = 0.05 \\
      \hline
      $\alpha=0.00001$ & 0.83 & 0.79 & 0.76 & 0.75 & 0.78 \\
      \hline
      $\alpha=0.00005$ & 0.82 & 0.80 & 0.77 & 0.78 & 0.75 \\
      \hline
      $\alpha=0.0001$  & 0.84 & 0.81 & 0.80 & 0.73 & 0.77 \\
      \hline
      $\alpha=0.0005$  & 0.85 & 0.83 & 0.84 & 0.81 & 0.79 \\
      \hline
      $\alpha=0.001$   & 0.84 & 0.84 & 0.82 & 0.78 & 0.76 \\
      \hline
    \end{tabular}
    \caption{Wpływ regularyzacji L2 ($\lambda$) i~normalizacji lokalnego kontrastu ($\alpha$) na~dokładność klasyfikacji
    sieci neuronowej}
    \label{table:wyniki1}
\end{table}

Całkowity czas badania wyniósł: 40 godzin 10 minut i 58 sekund.

\section{Omówienie wyników badań}
Wyniki uzysakne w~badaniu wskazują, że~wzraz ze~wzrostem wpływu regularyzacji~L2 na~sieć neuronową, dokładność
klasyfikacji ulegała pogorszeniu. Jednocześnie najlepsze wyniki osiągane były dla~wartości $\alpha$ w~okolicach 0.0005.

W~przypadku wpływu normalizacji lokalnego kontrastu wyniki są~zgodne z~oczekiwaniami, tj.~dla~odpowiednio dobranych
parametrów normalizacja ta~przynosi poprawę rezultatów. Sprzeczne z~przewidywaniami okazały się~zmiany wartości
dokładności dla~różnych wartości parametru $\lambda$. Spodziewano się, że~regularyzacja zapobiegając overfittingowi
polepszy dokładność klasyfikacji.

Wytłumaczeniem takiego stanu rzeczy może być brak wystąpienia zjawiska overfittingu, z~dwóch powodów:
\begin{itemize}
    \item zbyt mała liczba neuronów w~sieci, a~przez~to~niska szansa na~zbytnie dopasowanie się~sieci do~danych
          wejściowych,
    \item zbyt mała liczba iteracji podczas uczenia sieci.
\end{itemize}
Regularyzacja~L2 choć~zapobiega wystąpieniu overfittingu, to~jednak ma~negatywny wpływ na~szybkość uczenia się~sieci,
przez~co~potrzebna jest większa liczba iteracji w~procesie uczenia, aby~osiągnąć zadowalający poziom dokładności.

TODO badanie dla większej liczby iteracji (lepsze) lub dla większej liczby neuronów.

\chapter{Podsumowanie}
Choć początkowo sieci neuronowe nie~były zbyt szeroko stosowane, to~dzięki
opracowaniu metod głębokiego uczenia rozwój uczenia maszynowego znacznie przyspieszył. Obecnie metody
te~są wykorzystywane w~wielu mechanizmach, z~których dużej części używamy na~co~dzień. Celem mojej pracy
magisterskiej będzie opisanie tych mechanizmów oraz sprawdzenie jak~sprawdzają się~one w~różnych problemach
klasyfikacji.

\raggedright
\bibliography{tex/bibliography}{}
\bibliographystyle{unsrt}
% TODO move it to bibliography.bib
%{\begin{itemize}
%  \item \href{https://www.youtube.com/playlist?list=PL6Xpj9I5qXYEcOhn7TqghAJ6NAPrNmUBH}{Hugo Larochelle,
%  \item \href{http://deeplearning.net/tutorial/}{DeepLearning.net},
%  \item \href{https://www.coursera.org/course/ml}{Stanford University -~Machine Learning:\\
%  https://www.coursera.org/course/ml},
%  \item \href{http://deeplearning.net/demos/}{Deep Learning - Przykłady: http://deeplearning.net/demos},
%\end{itemize}
%



%
%JWTODO od tego momentu są luźne notatki
%
%CNN:
%Wstęp
%- można powiedzieć ogólnie o filtrach splotowych: na czym polega, kilka przykładów z obrazkami,
%Local connectivity:
%- nie łączymy neuronów ukrytych ze wszystkimi wejściami tylko z wybranym regionem
%  (bo tak, to by było za dużo wag do nauczenia),
%- neuron jest połączony ze wszystkimi kanałami,
%Parameters sharing:
%- niektóre neurony (różne ramki) mają dokładnie te same wagi,
%- zestawy ramek, które mają te same wagi tworzą ,,feature map'',
%- każdy kolor ma swoją macierz wag,
%- zmniejsza liczbę parametrów,
%- szuka tej samej cechy w różnych miejscach,
%- obrazek z 9.3 10:00,
%Dyskretny splot:
%- zero-padding: ramka dla obrazka z samych zer,
%Pooling/subsampling hidden units:
%- przeskalowanie obrazu: np. obraz dzielony na kwadraciki 2x2, z każdego wybierany jeden piksel
%(max,avg,itp.); otrzymujemy obraz 4 razy mniejszy,
%- zmniejsza liczbę wejść do następnej warstwy ukrytej,
%Całość:
%- obrazek z 9.6 3:00
%
%
%
%
%
%Deep Belief Network:
%- można rozpatrywać jako złożenie wielu RBMów jeden na drugim,
%- służy do generowania danych wejściowych,
%- mechanizm uczenia tej sieci zapoczątkował deep learning,
%- najpierw pre-training: warstwa po warstwie, modelowanie rozkładu prawdopodobieństwa,
%- wzorki z 7.8 (pierwsza minuta);
%
%DBN - Variational Bound:
%- funkcja log jest wklęsła,
%- zawsze średnia dwóch elementów z wykresu funkcji tworzy odcinek pod wykresem log,
%- stosujemy, gdy ciężko policzyć p(v) (estymujemy przez q(v)),
%- q(v) <= p(v), dla każdego v,
%- 
% 
%Kwantowe komputery w deep-learningu (normalnie uczenie jest trudne, bo~opiera
%się na~skomplikowanych operacjach probabilistycznych; takie operacje są
%wykonywane na komputerach kwantowych bardzo szybko). Wówczas każdy kubit
%reprezentuje funkcję prawdopodobieństwa dla~danego neuronu, a~połączenia między
%neuronami są dosłownie połączeniami pomiędzy kubitami (Google już pracuje nad
%wykorzystaniem komputerów kwantowych w deep learningu).
% 
%Stacked RBM/Stacked Autoencoder
%
%
%
%Tabelka z 7.4 (5:15):
%- pretraining pomaga,
%
%Spis treści:
%- podstawowe definicje, oznaczenia,
%- trochę o deep learning,
%- jak oceniać jakość mechanizmow,
%- opis różnych metod,
%- jak dobierać parametry sieci,
%- badania:
%  - rozpoznawanie emocji/obrazów/mowy/dźwięków \ldots,
%  - jakie typy sieci do jakich zastosowań,
%  - jakieś tabelki,
%  - opracowanie wyników,
%- podsumowanie.
%
%
%
%\begin{itemize}
%	\item ograniczona maszyna Boltzmana (RBM),
%	\item każda warstwa to RBM (uczymy po kolei, bez nadzoru), a na koniec
%\end{itemize}
%
%
%Metryki jakości rozpoznawania:
%\begin{itemize}
%  \item macierz przekłamań (słabe bo za dużo klas)
%  \item te wskaźniki co u Piotrka jakoś dopasować
%\end{itemize}
\end{document}
