\chapter{Badania}

\section{Plan badań}
% sieć bez ulepszeń
% sieć z weight decay
% sieć z learning rate decay
% sieć ze weight decay i learning rate decay
TODO opisać że badany będzie weight decay i rozmiar batcha (artykuł od Maćka)

\section{Środowisko sprzętowe}
Wszystkie badania zostały wykonane z wykorzystaniem następującego zestawu komputerowego:
\begin{itemize}
    \item \textbf{procesor}~--~TODO,
    \item \textbf{płyta główna}~--~TODO,
    \item \textbf{karta graficzna}~--~TODO,
    \item \textbf{pamięć RAM}~--~TODO.
\end{itemize}

\section{Wyniki badań}
TODO
% accuracy na zbiorze ewaluacyjnym
%    dla różnych weight decay
%    dla różnych learning rate decay
%    dla sieci z weight decay i learning rate decay (razem) o najlepszych parametrach spośród badanych
% accuracy na zbiorze ewaluacyjnym i testowym dla sieci w finalnej wersji (weight decay i learning rate decay)

\subsection{Omówienie wyników}
TODO
% dlaczego weight decay pomaga
% dlaczego learning rate decay pomaga
