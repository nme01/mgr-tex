\chapter{Badania}
\section{Wykorzystywana miara jakości klasyfikatora}
Podstawowymi miarami stosowanymi w~ocenie jakości sieci neuronowych są:
\begin{itemize}
    \item dokładność (\textit{ang.~accuracy}),
    \item średnia precyzja (\textit{ang.~average precision}).
\end{itemize}

\paragraph{Dokładność} wskazuje jaka część przykładów ze~zbioru ewaluacyjnego została sklasyfikowana poprawnie.
\begin{equation*}
    ACC = \frac{CP}{T}
\end{equation*}
gdzie $CP$ to~liczba poprawnie sklasyfikowanych przykładów, a~$T$ to~całkowita liczba przykładów.

Jeśli wśród przykładów istnieje spora nadreprezentacja przykładów jednej klasy wówczas wskaźnik ten~może być
niemiarodajny. Przykładowo: klasyfikator mający stwierdzać płeć wśród studentów Wydziału Elektroniki i~Technik
Informacyjnych Politechniki Warszawskiej mógłby wsakzywać dla~każdej osoby płeć męską, a~jego dokładność byłaby
dość wysoka. Dlatego w~takich przypadkach badając jakość sieci należy posłużyć się~również dodatkowymi miarami.

\paragraph{Precyzja} to~miara, która liczona jest dla~każdej z~klas z~osobna i~wyraża ile~przykładów sklasyfikowanych
 jako należące do~danej~klasy faktycznie należy do~tej klasy (przykładowo: jaka część obrazów sklasyfikowanych jako
 obraz z~psem faktycznie jest obrazem przedstawiającym to~zwierzę). W~ocenie jakości sieci neuronowych rozpoznających
 więcej niż~dwie klasy stosuje się~uśrednioną precyzję, tj.~średnią z~precyzji uzyskanych dla~każdej z~klas.
 \begin{equation*}
    PPV = \frac{TP}{TP+FP}
\end{equation*}
gdzie $TP$ to~prawidłowo rozpoznane przykłady danej klasy, a~$FP$ to~przykłady niepoprawnie sklasyfikowane jako
 należące do~tej klasy.

\section{Architektura sieci neuronowej}
Podczas tworzenia splotowej sieci neuronowej, należy dobrać wiele hiperparametrów takich jak:
\begin{itemize}
    \item liczba warstw splotowych,
    \item liczba jąder stosowanych do~wykonywania splotów w~każdej z~warstw splotowych,
    \item rozkład warstw typu max-pooling (\ref{sec:inferencja}),
    \item rozkład warstw normalizujących (\ref{sssec:normalizacja_odpowiedzi}),
    \item współczynnik uczenia (\ref{ssec:backpropagation}).
\end{itemize}

Ogólne zasady dotyczące pierwszych dwóch z~wymienionych punktów zostały opisane w~artykule
,,Rethinking the Inception Architecture for Computer Vision''(\cite{RIACV}). Posiłkując się~przytoczoną pracą można
wymienić kilka wskazówek przydatnych przy~ustalaniu hiperparametrów sieci:
\begin{itemize}
    \item unikanie zbyt małej liczby neuronów w~warstwach, w~sczególności w~warstwach początkowych. Warto zastosować
          kilkukrotność/kilkunastokrotność spodziewanej liczby klas, które ma~rozpoznawać sieć
          (w~przypadku CIFAR-10 jest to~10 klas). Warstwy końcowe mogą zawierać mniejszą liczbę neuronów, niż warstwy
          poprzednie.
    \item zmniejszenie rozmiaru danych wejściowych poprzez zastosowanie metod, takich jak:
          \begin{itemize}
              \item usunięcie brzegów, gdyż~zwykle zawierają mało istotne dane,
              \item zmniejszenie rozmiaru obrazka poprzez zastosowanie sklaowania.
          \end{itemize}
    \item używanie niewielkich filtrów splotowych (np. 3x3 lub 5x5 zamiast 7x7). Lepsze efekty daje zastosowanie dwóch
          warstw splotowych o~maskach 3x3 niż jednej maski 7x7,
    \item warto zacząć od 2 do 5 warstw splotowych (tyle samo warstw skalujących i~normalizujących), następnie zwiększać
          liczbę masek używanych w~warstwach splotowych na~przemian ze~zwiększaniem liczby warstw.
\end{itemize}

\subsection{Architektura badanej sieci}
Badana sieć w~swojej podstawowej wersji bazuje na~architekturze AlexNet przedstawionej w~artykule \cite{AlexNet}.
Po dokonaniu drobnych modyfikacji w~końcowych etapach przetwarzania obrazu, sieć składa się~z~następujących warstw:
\begin{enumerate}
    \item Warstwy splotowej z~64 maskami o rozmiarze 5x5x3 (wysokość x szerokość x liczba objętych kanałów).
          Maska przesuwana jest zawsze o~1~piksel (w~kierunku pionowym i~poziomym).
    \item Warstwy skalującej typu max-pooling o~wielkości filtra 3x3x1 (wysokość x szerokość x liczba objętych kanałów).
          Filtr jest przesuwany o~2~piksele (w~kierunku pionowym i~poziomym)
    \item Warstwy normalizującej (normalizacja lokalnej odpowiedzi).
    \item Warstwy splotowej z~64 maskami o rozmiarze 5x5x64 (wysokość~x~szerokość~x~liczba objętych kanałów).
          Maska przesuwana jest zawsze o~1 piksel (niezależnie od~kierunku przesuwania maski).
    \item Warstwy normalizującej (normalizacja lokalnej odpowiedzi).
    \item Warstwy skalującej typu max-pooling o~wielkości filtra 3x3x1 (wysokość~x~szerokość~x~liczba objętych kanałów).
          Filtr jest przesuwany o~2~piksele (w~kierunku pionowym i~poziomym).
    \item Warstwy w~pełni połączonej (standardowa warstwa w~sieciach neuronowych) z~384 neuronami i~funkcją aktywacji
          typu ReLU.
    \item Warstwy w~pełni połączonej z~192 neuronami i~funkcją aktywacji
          typu ReLU.
    \item Warstwy wyjściowej (również w~pełni połączonej) z~10 neuronami (tyle samo, co~klas do~rozpoznawania).
          Warstwa wyjściowa zawiera funkcję aktywacji typu softmax.
\end{enumerate}

\subsubsection{Przetwarzanie wstępne}
Dane wejściowe przed~tym, jak~trafią do~sieci neuronowej, poddawane są~przetwarzaniu wstępnemu. Sprowadza się~ono
do~przycięcia obrazka, tak~by~jego rozdzielczość wyniosła 24x24 piksele (oryginalna: 32x32 piksele). W~procesie
uczenia obrazek przycinany jest losowo, a~w~przypadku obrazka klasyfikowanego~--~wybierany jest środkowy fragment.
Dodatkowo, jeśli obrazek ma~być wykorzystywany w~procesie uczenia, poddawany jest on~zniekształceniu.

Po~wczytaniu i~przycięciu obrazka (w~przypadku uczenia: również po~zastosowaniu zniekształcenia) każdy z~obrazów jest
normalizowany (niezależnie od~innych). Normalizacja polega na~zapewnieniu, że~subpiksele w~każdym kanale
przyciętego obrazka (czerwonym, zielonym i~niebieskim) mają średnią wartość równą zero i~odchylenie standardowe równe~1.

% TODO dodać obrazek przedstawiający graf operacji
% TODO ustawienie seeda do generatora liczb losowych w tensorflow -- tf.set_random_seed(1)
%      http://stackoverflow.com/questions/36288235/how-to-get-stable-results-with-tensorflow-setting-random-seed


\section{Plan badań}
Celem niniejszej pracy naukowej jest sprawdzenie jak istotny wpływ na~dokładność klasyfikacji mają 2~czynniki:
\begin{itemize}
    \item regularyzacja typu L2 (\ref{sssec:reg_L2}),
    \item normalizacja lokalnej odpowiedzi (\ref{sssec:normalizacja_odpowiedzi}).
\end{itemize}

W~artykule ,,Practical Recommendations for Gradient-Based Training of Deep Architectures''
(\cite{practical-gradient-based}) skrótowo omówiono problem doboru hiperparametrów sieci. Jednym ze~sposobów
przedstawionych w~opracowaniu jest określenie wartości brzegowych dla~optymalizowanych hiperparametrów,
a~następnie zbadanie przestrzeni między. Po~zbadaniu zachowania sieci dla~tej przestrzeni hiperparametrów, można podjąć
decyzję o~przyjęciu jednego z~zestawów hiperparametrów dla~sieci lub~rozszerzyć pole poszukiwań o~kolejne obszary.

Dla~parametru regularyzacji~L2 (tzw.~\textbf{weight decay}) jako~górne ograniczenie przyjęto początkowo wartość~0.05.
Wybór bazował na~tym, że~w~podobnych sieciach, tj.~przeznaczonych do~identyfikacji obiektów przedstawianych na~obrazkach
z~bazy ImageNet (\cite{imagenet}), wartość tego hiperparametru nie~przekraczała~0.03. Badanie miało sprawdzić
również jak~sieć zachowywałaby~się~bez regularyzacji wag. Stąd jako dolne ograniczenie przyjęto wartość~0.

Dla~parametru decydującego o~wpływie normalizacji lokalnej odpowiedzi (tzw.~parametr $\alpha$) jako ograniczenie górne
przyjęto początkowo wartość~0.001, a~jako ograniczenie dolne wartość~0. Usprawiedliwienie dla~tych decyzji było
takie samo, jak~dla~wyborów dokonanych przy~hiperparametrze regularyzacji~L2.

Dla~każdej pary parametrów sieć była uczona 100,000 mini-zestawów danych (tzw.~\textbf{minibatch}), z~których
każdy zawierał 128 przykładów uczących. Po~każdym kroku uczenia pojedynczym mini-zestawem danych, sprawdzano
dokładność sieci na~zbiorze testowym. Po~wykonaniu wszystkich kroków uczenia dla~danej pary hiperparametrów
brano średnią dokładność sieci na~zbiorze testowym ze~100 ostatnich kroków uczenia. Wyniki przedstawiono w~tabelce
(\ref{table:wyniki1}).

\section{Wyniki badań}
\begin{table}[H]
    \centering
    \begin{tabular}{|l|l|l|l|l|l|}
      \hline
                       & $\lambda$ = 0.0005 & $\lambda$ = 0.001 & $\lambda$ = 0.005 & $\lambda$ = 0.01 & $\lambda$ = 0.05 \\
      \hline
      $\alpha=0.00001$ & 0.83 & 0.79 & 0.76 & 0.75 & 0.78 \\
      \hline
      $\alpha=0.00005$ & 0.82 & 0.80 & 0.77 & 0.78 & 0.75 \\
      \hline
      $\alpha=0.0001$  & 0.84 & 0.81 & 0.80 & 0.73 & 0.77 \\
      \hline
      $\alpha=0.0005$  & 0.85 & 0.83 & 0.84 & 0.81 & 0.79 \\
      \hline
      $\alpha=0.001$   & 0.84 & 0.84 & 0.82 & 0.78 & 0.76 \\
      \hline
    \end{tabular}
    \caption{Wpływ regularyzacji L2 ($\lambda$) i~normalizacji lokalnego kontrastu ($\alpha$) na~dokładność klasyfikacji
    sieci neuronowej}
    \label{table:wyniki1}
\end{table}

Całkowity czas badania wyniósł: 40 godzin 10 minut i 58 sekund.

\section{Środowisko sprzętowe}
Badania zostały wykonane z wykorzystaniem następującego zestawu komputerowego:
\begin{itemize}
    \item \textbf{procesor}~--~Intel Core i7-4771 3,5Ghz (8 rdzeni),
    \item \textbf{płyta główna}~--~TODO,
    \item \textbf{karta graficzna}~--~MSI GeForce GTX 780 Ti,
    \item \textbf{pamięć RAM}~--~TODO.
\end{itemize}

\section{Omówienie wyników badań}
Wyniki uzysakne w~badaniu wskazują, że~wzraz ze~wzrostem wpływu regularyzacji~L2 na~sieć neuronową, dokładność
klasyfikacji ulegała pogorszeniu. Jednocześnie najlepsze wyniki osiągane były dla~wartości $\alpha$ w~okolicach 0.0005.

W~przypadku wpływu normalizacji lokalnego kontrastu wyniki są~zgodne z~oczekiwaniami, tj.~dla~odpowiednio dobranych
parametrów normalizacja ta~przynosi poprawę rezultatów. Sprzeczne z~przewidywaniami okazały się~zmiany wartości
dokładności dla~różnych wartości parametru $\lambda$. Spodziewano się, że~regularyzacja zapobiegając overfittingowi
polepszy dokładność klasyfikacji.

Wytłumaczeniem takiego stanu rzeczy może być brak wystąpienia zjawiska overfittingu, z~dwóch powodów:
\begin{itemize}
    \item zbyt mała liczba neuronów w~sieci, a~przez~to~niska szansa na~zbytnie dopasowanie się~sieci do~danych
          wejściowych,
    \item zbyt mała liczba iteracji podczas uczenia sieci.
\end{itemize}
Regularyzacja~L2 choć~zapobiega wystąpieniu overfittingu, to~jednak ma~negatywny wpływ na~szybkość uczenia się~sieci,
przez~co~potrzebna jest większa liczba iteracji w~procesie uczenia, aby~osiągnąć zadowalający poziom dokładności.

TODO badanie dla większej liczby iteracji (lepsze) lub dla większej liczby neuronów.
