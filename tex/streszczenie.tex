\newpage
\thispagestyle{empty} % remove page numbering
\section*{Streszczenie}
\paragraph{Tytuł:} Metody głębokiego uczenia w~wybranych problemach klasyfikacji \\\\
Celem niniejszej pracy było zbadanie różnych metod mających na~celu poprawienie jakości klasyfikacji. We~wstępie
pracy przedstawiono historię rozwoju sztucznej inteligencji oraz jej~współczesne zastosowania. W~kolejnym rozdziale
omówiono budowę i~zasady działania warstwowych sieci neuronowych oraz zaprezentowano ich~podstawowe rodzaje.
W~rozdziale trzecim omówiono szczególny przypadek sieci neuronowych, jakim są sieci splotowe oraz zaprezentowano
metody stosowane w~celu poprawy jakości klasyfikacji. Następnie przedstawiono narzędzia programistyczne
wykorzystywane przy~budowie mechanizmów sztucznej inteligencji. Rozdział piąty omawia eksperymenty, jakie zostały
przeprowadzone w~celu zbadania wpływu różnych zabiegów na~jakość klasyfikacji obrazów z~bazy CIFAR-10.

\paragraph{Słowa kluczowe:} sztuczna inteligencja, uczenie maszynowe, sztuczne sieci neuronowe, sieci splotowe,
rozpoznawanie obrazów, regularyzacja, lokalna normalizacja odpowiedzi

\newpage
\thispagestyle{empty} % remove page numbering
\section*{Abstract}
\paragraph{Title:} Methods of~Deep Learning in~chosen classification problems \\\\
The~main purpose of~the~thesis was to~examine different methods used for the~improvement of~classification's quality.
In~the~introduction history of~development of~Artificial Intelligence is~presented as~well as~its~current applications.
Next chapter presents multi-layer neural networks and their basic types. In~the~third chapter special type of~a~neural
network is presented: a~convolutional neural network. Also the~chapter describes methods used for improvement
of~classification's quality. After that, software tools used for~artificial intelligence development are presented.
The~fifth chapter describes experiments condcuted in~order to~examine the~influence of~different methods on~a~quality
of~classification of~CIFAR-10 images.

\paragraph{Keywords:} artificial intelligence, machine learning, artificial neural networks, convolutional networks,
image recognition, regularization, local response normalization