\chapter{Opis środowiska}
W celu uniknięcia konieczności zajmowania się niskopoziomowymi problemami, takimi jak:
\begin{itemize}
    \item zrównoleglanie obliczeń wektorowych,
    \item wykonywanie obliczeń na~karcie graficznej,
    \item liczenie gradientów operacji wykonywanych przez sieć neuronową (takich jak~sploty, normalizacje, aktywacje),
    \item wczytywanie danych,
    \item itp.
\end{itemize}
postanowiono skorzystać z~jednej~z~dostępnych bibliotek, która zajmuje się~owymi zagadnieniami. Pozwala to~na~skupienie
się~na~architekturze sieci oraz na odpowiednim doborze operacji mających na~celu zapewnienie jak~najlepszej
klasyfikacji.

\section{Wybrana biblioteka}
Do~realizacji badań wykonywanych w~ramach niniejszej pracy magisterskiej wykorzystano bibliotekę
\textbf{Tensorflow} firmy Google.

W~rozdziale opisano podstawowe zagadnienia dotyczące wybranego narzędzia. Wiedza ta~pozwoli na~zrozumienie
informacji o~architekturze sieci przedstawianych w~dalszych częściach pracy.

\section{Graf operacji}
Tworzenie sztucznej sieci neuronowowej przy~użyciu wymienionej powyżej biblioteki, można podzielić na~dwa etapy:
\begin{enumerate}
    \item zdefiniowanie grafu operacji,
    \item wykonanie wybranych operacji z~grafu.
\end{enumerate}

Graf operacji definiuje w~jaki sposób dane wejściowe mają być przetwarzane po to, by osiągnąć rezultat na wyjściu.
Każda operacja może przyjmować pewne dane wejściowe. Dane wejściowe mogą być zarówno danymi wczytanymi z~dysku
czy~z~klawiatury. Najczęściej jednak są one~wynikami innych operacji. W~ten sposób operacje tworzą wzajemne zależności,
gdzie wyjście jednej z~nich jest jednocześnie wejściem innej. Operacje mogą być ze sobą grupowane poprzez tworzenie
tzw.~zakresów (\textit{ang.~scope}). Przykładowy graf przedstawiono na rysunku \ref{img:tf-smpl-grf}.

\begin{figure}[H]
	\centering
	\includegraphics[width=0.5\linewidth]{img/tf-sample-graph.jpg}
	\caption{Przykładowy graf operacji}
	\label{img:tf-smpl-grf}
\end{figure}
