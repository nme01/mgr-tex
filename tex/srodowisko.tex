\chapter{Opis środowiska}
W celu uniknięcia konieczności zajmowania się niskopoziomowymi problemami, takimi jak:
\begin{itemize}
    \item zrównoleglanie obliczeń wektorowych,
    \item wykonywanie obliczeń na~karcie graficznej,
    \item liczenie gradientów operacji wykonywanych przez sieć neuronową (takich jak~sploty, normalizacje, aktywacje),
    \item wczytywanie danych,
    \item itp.
\end{itemize}
postanowiono skorzystać z~jednej~z~dostępnych bibliotek, która zajmuje się~owymi zagadnieniami. Pozwala to~na~skupienie
się~na~architekturze sieci oraz na odpowiednim doborze operacji mających na~celu zapewnienie jak~najlepszej
klasyfikacji.

\section{Wybrana biblioteka}
Do~realizacji badań wykonywanych w~ramach niniejszej pracy magisterskiej wykorzystano bibliotekę
\textbf{Tensorflow} firmy Google.

W~rozdziale opisano podstawowe zagadnienia dotyczące wybranego narzędzia. Wiedza ta~pozwoli na~lepsze rozumienie
informacji o~architekturze sieci przedstawianych w~dalszych częściach pracy.
