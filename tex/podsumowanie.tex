\chapter{Podsumowanie}
W ciągu ostatnich 10 lat dzięki znacznemu wzrostowi mocy obliczeniowej, szczególnie dzięki rozwojowi takich technologii
jak~Nvidia CUDA (\cite{nvidia-cuda}), istotnie wzrosło zainteresowanie sztucznymi sieciami neuronowymi. Znalazły
one~praktyczne zastosowanie w~problemach takich jak:
\begin{itemize}
    \item rozpoznawanie mowy,
    \item rozpoznawanie obrazów,
    \item generowanie sygnałów dźwiękowych,
    \item generowanie danych graficznych (np.~pisma ręcznego),
    \item przetwarzanie języka naturalnego.
\end{itemize}

Między innymi dużym zainteresowaniem cieszą się~mechanizmy służące do~rozpoznawania obrazów takie jak~splotowe sieci
neuronowe. Przetwarzanie obrazów jest ważnym zagadnieniem na~drodze do~stworzenia tzw.~sztucznej inteligencji ogólnego
przeznaczenia (ang.~general purpose artificial intelligence, \cite{strong-AI}). Prace nad~stworzeniem takiego mechanizmu
trwają. Zagadnienie to~jest badane m.in. przez~firmę Google w~ramach projektu DeepMind. Mechanizm rozwijany
przez~przedsiębiorstwo z~Mountain View wykorzystuje uczenie ze~wzmocnieniem wraz ze~splotowymi sieciami neuronowymi.
Stąd, można powiedzieć, że~prace nad~splotowymi sieciami neuronowymi zbliżają ludzkość do~zbudowania uniwersalnego
mechanizmu potrafiącego rozwiązywać zagadnienia z~różnych dziedzin, podobnie jak~człowiek.

Rozpoznawanie obrazów pełni znaczącą rolę nawet u~większości zwierząt, o~czym może świadczyć to,~że~ośrodek wzrokowy
u~ssaków jest jednym z~najbardziej złożonych. Przypuszczać można, iż~wynika to~z~tego, że~najwięcej informacji
z~otaczającego świata można pozyskiwać z~sygnałów optycznych. O~tym jak~istotne (a~jednocześnie skomplikowane) jest
to~zagadnienie może świadczyć fakt, iż~prawie połowa kory nowej u~naczelnych odpowiada za~przetwarzanie sygnałów
wzrokowych (\cite{primate-cerebral-cortex}).

Obecnie splotowe sieci neuronowe są~przedmiotem wielu badań oraz znajdują zastosowanie w~wielu problemach. Istnieją
nawet portale~ogłaszające konkursy, w~których dla~twórców najlepszych mechanizmów przewidziane są~zwykle wysokie nagrody
pieniężne (np. Kaggle, \cite{kaggle-competitions}). Splotowe sieci neuronowe zdobywają większość nagród w~konkursach
dotyczących rozpoznawania obrazów.

W~niniejszej pracy magisterskiej badane sieci miały dość prostą architekturę, dzięki czemu ich uczenie nie~wymagało
zastosowania wysokich mocy obliczeniowych oraz nie~pochłaniało dużych ilości czasu. Dzięki temu możliwe było
przeprowadzenia wielu badań w~stosunkowo krótkim czasie (zwykle pojedyncze badanie nie~trwało dłużej niż~4~dni).
Opisana sieć może być wykorzystana w~przyszłości do~badań nad~nowymi metodami poprawiającymi zarówno dokładność
klasyfikacji (np.~dropout, warstwy typu batch normalization), jak~i~jej szybkość (np.~zmodyfikowane wersje
zaprezentowanej architektury, inne modele neuronów, różne rozmiary wsadu z~przykładami w~procesie uczenia).

W~zastosowaniach profesjonalnych (np.~prace wygrywające konkursy na~stronie Kaggle) zwykle stosowane są~bardzo złożone
architektury, do~uczenia których wykorzystywane są~komputery z~wieloma (np.~czterema) kartami graficznymi. Sam proces
uczenia takich mechanizmów zwykle liczony jest w~tygodniach lub~miesiącach (a~nie pojedyncznych dniach, jak~w~przypadku
sieci opisywanej w~ramach niniejszej pracy magisterskiej). Jednakże owa~sieć może być rozbudowana o~dodatkowe warstwy
i~dodatkowe mechanizmy. Ponadto została ona~zaimplementowana w~taki sposób, by~obliczenia wykonywane w~procesie uczenia
mogły wykorzystywać wiele kart graficznych.

Rozbudowana wersja sieci, wykorzystująca 15~warstw i~uczona przy~użyciu czterech kart graficznych, zostanie zgłoszona
do~konkursu ogłoszonego przez~The~Nature Conservacy (\cite{nature-conservacy}). Problem opublikowany na~stronie Kaggle
dotyczy rozpoznawania na~zdjęciach różnych gatunków ryb (m.in.~różnych gatunków tuńczyków i~rekinów) złowionych
przez~łodzie rybackie na~Zachodnim i~Środkowym Pacyfiku.